% !TeX program = pdfLaTeX
\documentclass[12pt]{article}
\usepackage{amsmath}
\usepackage{graphicx,psfrag,epsf}
\usepackage{enumerate}
\usepackage{natbib}
\usepackage{textcomp}
\usepackage[hyphens]{url} % not crucial - just used below for the URL
\usepackage{hyperref}
\providecommand{\tightlist}{%
  \setlength{\itemsep}{0pt}\setlength{\parskip}{0pt}}

%\pdfminorversion=4
% NOTE: To produce blinded version, replace "0" with "1" below.
\newcommand{\blind}{0}

% DON'T change margins - should be 1 inch all around.
\addtolength{\oddsidemargin}{-.5in}%
\addtolength{\evensidemargin}{-.5in}%
\addtolength{\textwidth}{1in}%
\addtolength{\textheight}{1.3in}%
\addtolength{\topmargin}{-.8in}%

%% load any required packages here



% Pandoc citation processing

\usepackage{amssymb}
\usepackage{amsmath}

\begin{document}


\def\spacingset#1{\renewcommand{\baselinestretch}%
{#1}\small\normalsize} \spacingset{1}


%%%%%%%%%%%%%%%%%%%%%%%%%%%%%%%%%%%%%%%%%%%%%%%%%%%%%%%%%%%%%%%%%%%%%%%%%%%%%%

\if0\blind
{
  \title{\bf Data Augmentation with Polya-Gamma Latent Variables for
Logistic Models}

  \author{
        Shen Qu \thanks{STAT 501 Statistical Literature and Problems;
The Fariborz Maseeh Department of Mathematics + Statistics} \\
    Portland State University\\
     and \\     Supervisor Prof.~Daniel Taylor-Rodriguez \\
    Portland State University\\
      }
  \maketitle
} \fi

\if1\blind
{
  \bigskip
  \bigskip
  \bigskip
  \begin{center}
    {\LARGE\bf Data Augmentation with Polya-Gamma Latent Variables for
Logistic Models}
  \end{center}
  \medskip
} \fi

\bigskip
\begin{abstract}
In Bayesian Statistics, Data Augmentation (DA) methods can generate the
auxiliary or latent variables to solve some questions that cannot sample
directly or the probability density/mass functions doesn't have a closed
form. This paper introduces the basic DA approaches and several
examples. For binary response, both probit models and logitstic model
can use Gibb's samplers to simulate the target distribution through the
single-layer latent variables. For logitstic regression models,
generating the posterior distribution from the Pólya--Gamma family is a
fast, exact simulation method.
\end{abstract}

\noindent%
{\it Keywords:} Data Augmentation, Logistic Models, Polya-Gamma
\vfill

\newpage
\spacingset{1.45} % DON'T change the spacing!

\hypertarget{introduction}{%
\section{Introduction}\label{introduction}}

Data Augmentation (DA) is a technique of newly creating synthetic data
from existing data. In Statistics contest, DA algorithm includes, but
not limited to Markov chain Monte Carlo (MCMC) method. There are plenty
of mathematical and computational techniques such as EM algorithm which
can fertilize the DA field exploration. In Deep Learning field, the DA
techniques add some slightly modified copies of already existing data
for increasing the amount of data \citep{shortenSurveyImageData2019}.
This paper focus on the DA applications of MCMC in Bayesian statistics.
It introduces the basic ideas of MCMC algorithm through several examples
and explain its main properties and conditions, that is constructing a
successes DA Markov chain requires the condition of \emph{Harris
ergodic}, which includes three properties: irreducible, aperiodic, and
Harris recurrent.

These approaches can be applied on the binary response regression
models. As a type of Generalized Linear Models (GLMs), the response is
treated as the results of a Bernoulli process. Using a link function of
probit or logit, one can construct a model with the dependent binary
variable \(\mathbf{Y}\), the vectors of covariates \(\mathbf{X}\) and
parameters \(\boldsymbol{\beta}\).

There are many approaches fitting these models through the Gibb's
samplers. From literature, we find two simple simulation methods with
some single-layer latent variables for the probit and logistic
regression models. In probit version, the Bayesian methods generate the
location-mixture latent variables from two truncated normal
distributions. For logistic models, the latent variables is drawn from
the Polya-Gamma family. Hence, its posterior distribution is a scale
mixture with the linear part.

\hypertarget{literature}{%
\section{Literature}\label{literature}}

A main goal of Bayesian approach is to identify the posterior
distribution. There are three main methods to estimate the target
posterior density of \(\beta\): asymptotic expansions, numerical
integration, and Monte Carlo integration. When the sample size is small,
asymptotic expansions may not work well. When the canonical exponential
families with normal or conjugate priors are concave on the log scale,
the numerical method can converge quickly to estimate the posterior
joint or marginal distribution \citep{ohaganKendallAdvancedTheory2004}.
In many cases, the numerical integration is difficult for the
high-dimension models. \citet{zellnerBayesianAnalysisDichotomous1984}
proposed an importance sampling generalization of Monte Carlo
simulation. They found that Monte Carlo methods are more adequate and
efficient to achieve a good approximation.
\citet{gelfandSamplingBasedApproachesCalculating1990} develop these
methods to Gibbs sampling. When sampling from the conditional
distribution is easier, Gibbs sampler is a highly useful MCMC method to
sample from multivariate distributions. Metropolis-Hastings algorithm
\citep{tierneyMarkovChainsExploring1994} is another important progress
of MCMC. It is the generalized version of Gibb's sampler. A systematic
review can be found in \citet{agrestiBayesianInferenceCategorical2005}
and \citet{agrestiCategoricalDataAnalysis2012}.

For binary response, the generalized linear models use a link function
such as probit or logit to connect the linear part to the response.
\citet{albertBayesianAnalysisBinary1993} provide a computational simple
Bayesian approach for probit regression modeling. They assume the
relationship between the binary data and the latent variables. The
question can be solved through an underlying normal regression model.
The models have a hierarchical prior structure as follow:

\[Z\sim N_p(\boldsymbol{X\beta,I}),\quad \boldsymbol{\beta}\sim N_p(\boldsymbol{A\beta_0,\sigma^2I}), \quad (\beta_0,\sigma^2)\sim \pi(\beta_0,\sigma^2)\]

where \(Z\) is the latent variable, \(\boldsymbol{\beta}\) is the
parameter vector of the linear predictors. \(\boldsymbol{A\beta_0}\) is
the linear subspace, and assuming \(\boldsymbol{\beta_0,\sigma^2}\) are
independent and noninformative priors. This method can also extend to
ordered multinomial responses and mixture models
\citep{hoffFirstCourseBayesian2009, gelmanBayesianDataAnalysis2020}.

The Bayesian logistic models using approximation methods for quite some
time. \citet{piegorschEmpiricalBayesEstimation1996} suggest the
parametric empirical Bayes methods using marginal maximum likelihood
estimate hyperparameters and extend the link function.
\citet{greenlandPuttingBackgroundInformation2001} discuss that the
approximation may be inadequate with sparse data and they suggest to use
conjugate priors to conduct exact analysis. A new step is that
\citet{polsonBayesianInferenceLogistic2013} introduce an new method that
assume the latent variables follow a Polya-Gamma distribution. It is as
simple as \citet{albertBayesianAnalysisBinary1993} 's method, which
don't need multiple layers. This approach assign the linear part at the
scale parameter. \citet{choiPolyaGammaGibbsSampler2013} prove that this
polya-gamma Gibbs sampler for logistic model is uniformly ergodic. Some
further discussions of this DA algorithm include convergence rates of
the DA and the Haar PX-DA algorithm
\citep{choiConvergenceAnalysisGibbs2014}, a sandwich version of DA
\citep{choiComparisonTheoremData2016}, and the analysis of variance
\citep{choiAnalysisPolyaGammaGibbs2017}.

\hypertarget{appoaches}{%
\section{Appoaches}\label{appoaches}}

\subsection{Basic ideas}
\label{sec:basic}

\begin{itemize}
\tightlist
\item
  Expectation
\end{itemize}

Assume a random variable \(g(X)\) and pdf
\(f_{X}(x):\mathbb{R}^{p} \to [0,\infty)\). Denote interested
\(E[g(x)]\) is the expected value or mean of \(g(X)\)
\citep[\emph{Def2.2.1}]{casellaStatisticalInference2002}.

\[E_{f_{X}}[g(x)]= \int_{\mathbb{R}^{p}} g(x) f_{X}(x) dx\]

\begin{itemize}
\tightlist
\item
  Joint, conditional, and marginal density
\end{itemize}

A probability function with a bivariate random vector \(f(x,z)\) is
called joint pdf.

A marginal probability density function of X is the integral of joint
pdf \(f(x,z)\) on X: \(f_{X}(x)=\int_{\mathbb{R}^p} f(x,z)dz\).

The relationship among joint pdf, marginal pdf, and conditional pdf of
\(X\) given \(Z\) is: \(f(x,z)=f(x|z)f(z)\)

\begin{description}
\item[\textbf{Example 1}]
Given \(f_{X}(x)=3x^{2}I_{(0,1)}(x)\), then
\(E[X]=\int_{0}^{1} xf_{X}(x)dx=\left.\frac34x^4\right|_0^1=0.75\).
\end{description}

Suppose the joint pdf \(f(x,z)=3xI_{(0<z<x<1)}\), then we can get all
the rest of density functions:

\[\begin{aligned}
f_Z(z)&=\int_{z}^{1} f(x,z)dx =\frac32(1-z^2)\\
f_{Z|X}(z|x)&=\frac{f(x,z)}{f_Z(z)}=\frac1{x}I(0<z<x)\\
f_{X|Z}(x|z)&=\frac{f(x,z)}{f_X(x)}=\frac{2x}{1-z^2}I(z<x<1)\\
\end{aligned}\]

\begin{description}
\item[\textbf{Example 2}:]
Given a joint pdf
\(f(x,z)=(\sqrt{2}\pi)^{-1}\exp\left[-(x^2-\sqrt{2}xz+z^2)\right]\),
which is a bivariate normal density with means \(\mu_X=\mu_Z=0\),
variances \(\sigma_X=\sigma_Z=1\), and correlation \(\rho=2^{-1/2}\), we
can get \(f_{X}(x)\) by integral of \(f(x,z)\).
\end{description}

\[\begin{aligned}
f_{X}(x)&=\int_{-\infty}^\infty f_{X,Z}(x,z)\,\mathrm dz \\
&= \frac{\exp(-\frac12x^2)}{\sqrt{2\pi}}\int_{-\infty}^\infty \frac{1}{\sqrt{2\pi\frac12}}\exp\left[-\frac12(z-\frac{x}{\sqrt{2}})^2\right]\,\mathrm dz \\
&= \frac{1}{\sqrt{2\pi}}\exp(-\frac12x^2)
\end{aligned}\]

Another way is, suppose \(f(z|x)\sim N(\frac{x}{\sqrt{2}}\frac12)\), we
also can get
\(f_{X}(x)=\frac{f(x,z)}{f(z|x)}=\frac{1}{\sqrt{2\pi}}\exp{(-\frac12x^2)}\)

Then, we can utilize these properties to construct MCMC algorithm to
solve the more complex questions.

\begin{itemize}
\tightlist
\item
  Monte Carlo approximation
\end{itemize}

In some case, \emph{constructing estimates of features of probability
distributions are impossible or very difficult to compute analytically}

\emph{if we have data from an unknown probability distribution belonging
to a model we can estimate that distribution in some way.} \emph{if we
have a feature (parameter) of the true distribution we want to estimate,
use Monte Carlo to estimate the feature by random quantities generated
using the estimated distribution. }
\citep[ch.~10.1]{doksumMathematicalStatisticsBasic2015}

Regardless of the distribution, if we draw i.i.d.
\(g(X_{1}),g(X_{2}),\ldots,g(X_{m})\) from \(f_{X}(x)\), then, for large
\(m\), the empirical distribution of is arbitrarily close to the
distribution under\(f_{X}(x)\). That is
\(\frac{1}{m}\sum_{i=1}^{m} g(X_{i})\overset{a.s}{\to}E_{f_{X}}[g(x)]\).

\begin{itemize}
\tightlist
\item
  Markov Chain
\end{itemize}

When it is impossible to simulate from \(f_{X}(x)\), such as
\(f_{X}(x)\) doesn't have closed form or \(f(x,z)\) is hard to integral,
we can construct a Markov chain. Suppose
\(X = \{X_{i}\}^{\infty}_{i=0}\), with state space \(\mathbf{X}\). If
the current state of the chain is \(X_{i} = x\), then the density of the
next state, \(X_{n+1}\), is \(k(\cdot|x)\). \textbf{???}

\begin{itemize}
\tightlist
\item
  Markov Chain Monte Carlo (MCMC)
\end{itemize}

Markov Chain Monte Carlo is a homogeneous positive recurrent Markov
chain
\citep{tannerCalculationPosteriorDistributions1987, swendsenNonuniversalCriticalDynamics1987a}.
\(K(x,x')\) is called the transition kernel which has an unique
stationary density \(k(x'|x)\). The conditional density \(k(x'|x)\) is
also called the Markov transition density (Mtd).

Our goal is to find \(f_{X}(x)\). The conditions are that a joint pdf
\(f:\mathbb{R}^p\times\mathbb{R}^q \to [0,\infty)\) exists, which
includes a auxiliary real variable \(Z\). If the invariant x-marginal
pdf \(f_{X}(x)=\int_{\mathbb{R}^q} f(x,z)dz\) holds. And the associated
two conditional pdfs \(f_{X|Z}(x'|z)\) and \(f_{Z|X}(z|x)\) are
applicable for simulation.

We can shows that \(k(x'|x)\) is a pdf of a random variable \(X'\)

\[\begin{aligned}
\int_{X} k(x'|x)d x' &=\int_{X}\left[\int_{Z}f_{X|Z}(x'|z)f_{Z|X}(z|x) dz\right]dx'\\
&=\int_{Z}f_{Z|X}(z|x) \left[\int_X f_{X|Z}(x'|z)dx'\right]dz\\
&=\int_{Z}f_{Z|X}(z|x)dz=1
\end{aligned}\]

Even without the \(k(x'|x)\) in closed form, we still can simulate the
Markov chain \(X\) as follows.

\begin{quote}
Given an initial value \(x_0\)
\end{quote}

\begin{quote}
\begin{enumerate}
\def\labelenumi{\arabic{enumi}.}
\tightlist
\item
  Draw \(z^{(1)} \sim f_{Z |X}(\cdot|x^{(0)})\).
\end{enumerate}
\end{quote}

\begin{quote}
\begin{enumerate}
\def\labelenumi{\arabic{enumi}.}
\setcounter{enumi}{1}
\tightlist
\item
  Draw \(x^{(1)} \sim f_{X|Z} (\cdot|z^{(1)})\).
\end{enumerate}
\end{quote}

\begin{quote}
Repeat the two steps \ldots{}
\end{quote}

\begin{quote}
\begin{enumerate}
\def\labelenumi{\roman{enumi}.}
\tightlist
\item
  Draw \(z^{(i)} \sim f_{Z |X}(\cdot|x^{(i-1)})\).
\end{enumerate}
\end{quote}

\begin{quote}
i+1. Draw \(x^{(i)} \sim f_{X|Z} (\cdot|z^{(i)})\).
\end{quote}

\begin{quote}
At the end, we can calculate
\(E_{f_{X}}[g(x)]\approx\frac{1}{m}\sum_{i=1}^m g(X_i)\)
\end{quote}

\begin{description}
\item[\textbf{Example 3}]
The Student's \(t_4\) with four degree of freedom.
\end{description}

Assume we don't know the target
\(f_{X}(x)=\frac38(1+\frac{x^2}4)^{-\frac52}\). We can take
\(f(x,z)=\frac{4}{\sqrt{2\pi}}z^{\frac32}\exp\{-z(\frac{x^2}{2}+2)\} I_{(0,\infty)}(z)\),
which satisfied \(f_{X}(x)=\int_{\mathbb{R}^p} f(x,z)dz\). Then, it is
easy to draw from two conditional pdf of
\((Z|X=x) \sim Gamma(\frac52,\frac{x^2}2+2)\) and
\((X|Z=z) \sim N(0,z^{-1})\).

\begin{itemize}
\tightlist
\item
  Simple slice sample
\end{itemize}

The \textbf{simple slice sampler} is a DA technique through a
\emph{stepping-out and shrinkage} procedures
\citep{nealSliceSampling2003}. If we can factorize \(f_{X}(x)\) to
\(q(x)l(x)\), supposing an auxiliary variable \(Z\) satisfies
\((Z|X=x)\sim Uniform(0,l(x))\). The joint distribution over \(x\) and
\(z\) define a region \(\{(x,z):0\le z\le l(x)\}\) under the curve of
\(l(x)\). A random state of \(x^{(0)}\) can give a vertical ``slice''
for all possible \(z\) bounded from zero to \(l(x^{(0)})\). Then an
uniformly selected \(z^{(1)}\) can give a horizontal ``slice'' for all
\(x\) inside the curve. That slice is from zero to
\(\int_{z^{(1)}}^{l(x)} f_{X|Z}(x|z)dx\), the union of intervals that
all \(x\) satisfy \(l(x)\ge z^{(1)}\). The next state \(x^{(1)}\) around
\(x^{(0)}\) is accepted only if it doesn't step out the horizontal
``slice''. Each updating will shrink \((x,z)\) until the whole region is
explored.

\begin{description}
\item[Example 1 Cont.]
Rewrite \(f_{X}(x)=3x^{2}I_{(0,1)}(x)=[3x I_{(0,1)}(x)][x]=q(x)l(x)\).
\end{description}

For \(l(x)=x\), the vertical slice is \(\{z|x: 0\le z\le x \}\).
\((Z|X=x) \sim Unif(0,x)\).

The horizontal slice is \(\{x|z: x\ge z\}\). Given
\(f_{X|Z}(x|z)=\frac{2x}{1-z^2}I(z<x<1)\), The union of intervals is
\(\int_z^x f_{X|Z}(x|z)dx=\frac{x^2-z^2}{1-z^2}\). Then
\((X|Z=z)\sim Unif(0,\frac{x^2-z^2}{1-z^2})\)

Suppose \(U \sim Unif(0,1)\), we can get \((Z|X)=xU\) and
\((X|Z)=\sqrt{U(1-z^2)+z^2}\) by transformation.

\begin{quote}
Given an initial value \(x^{(0)}\)
\end{quote}

\begin{quote}
\begin{enumerate}
\def\labelenumi{\arabic{enumi}.}
\tightlist
\item
  Draw \(u^{(1)} \sim Unif(0,1)\)
\end{enumerate}
\end{quote}

\begin{quote}
\begin{enumerate}
\def\labelenumi{\arabic{enumi}.}
\setcounter{enumi}{1}
\tightlist
\item
  Calculate \(z^{(1)} =x^{(0)}u^{(1)}\)
\end{enumerate}
\end{quote}

\begin{quote}
\begin{enumerate}
\def\labelenumi{\arabic{enumi}.}
\setcounter{enumi}{2}
\tightlist
\item
  Calculate \(x^{(1)} =\sqrt{u^{(1)}(1-(z^{(1)})^2)+(z^{(1)})^2}\).
\end{enumerate}
\end{quote}

\begin{quote}
Update \((x^{(i)},z^{(i)})\) by repeating step 1, 2, 3, \ldots{}
\end{quote}

\begin{quote}
Calculate \(E_{f_{X}}[g(x)]\approx\frac{1}{m}\sum_{i=1}^m g(X_i)\)
\end{quote}

\hypertarget{properties}{%
\subsection{Properties}\label{properties}}

\hypertarget{conditions}{%
\subsubsection{Conditions}\label{conditions}}

A well-behaved Markov Chain should be Harris ergodic, which satisfies
three properties: irreducible, aperiodic, and recurrent. A sufficient
condition for Harris ergodicity is

\[\mathcal{K}: k(x'|x)>0\quad\forall x', x\in \mathbf{X}\]

\begin{itemize}
\tightlist
\item
  \textbf{Detailed balance}
\end{itemize}

For all \(x, x'\in \mathbf{X}\), let

\[\delta(x',x)=k(x'|x)f_{X}(x)=\int_{Z} \frac{f(x',z)}{f_Z(z)}\cdot\frac{f(z,x)}{f_X(x)}f_{X}(x) dz\]
\[\delta(x,x')=k(x|x')f_{X}(x')=\int_{Z} \frac{f(x,z)}{f_Z(z)}\cdot\frac{f(z,x')}{f_X(x')}f_{X}(x') dz\]

So \(\delta(x',x)=\delta(x,x')\) is a symmetric function satisfying

\[\delta(x,x')=f_X(x)k(x'|x)=f_X(x')k(x|x'), \forall x, x'\]

which is called the \emph{detailed balance} condition.

A type of MCMC, Metropolis Sampling Algorithm defines a symmetric Markov
kernel \(K_0(\cdot,\cdot)\) in the original Metropolis algorithm and

\[r(x,x')=\min\left(1,\frac{f_X(x')}{f_X(x)}\right)\]

If and only if \(r(x,x')=\frac{\delta(x,x')}{f_X(x)K_0(x,x')}\)

\[K(x,x')=r(x,x')K_0(x,x'),\quad x\neq x'\]

Then \(K\) is a Markov kernel satisfying detailed balance. The Markov
chain is reversible with respect to \(f_X(x)\). When the MCMC always
satisfying detailed balance \(r\equiv 1\), this algorithm is called
Gibbs' Sampler
\citep[ch.~10.4.3]{doksumMathematicalStatisticsBasic2015}.

\begin{itemize}
\tightlist
\item
  \textbf{Invariant (stationarity)}
\end{itemize}

\(f_{X}\) is an invariant density for \(K\) when

\[\int_X k(x'|x)f_{X}(x) dx= f_{X}(x')\]

which means that, if \(X_n\sim f_{X}\), then \(X_{n+1}\sim f_{X}\). The
Markov chain is time homogeneous.

For the case of the multivariate distribution
\(\boldsymbol{\beta}=(\beta_1,...,\beta_p)^T\), the Gibbs sampling
procedure requires the stationary density is an unique distribution
defined by the full conditionals, which mean each single component of
the random vector given the values of the rest.

\hypertarget{convergence}{%
\subsubsection{Convergence}\label{convergence}}

For a reversible Markov chain X satisfies Harris ergodicity, the
\textbf{Strong Law of Large Numbers} holds. The marginal density of
\(X_i\) will converge to the invariant density \(f_{X}\) no matter how
the chain is started. Denote \(P^{n}(x,A)=Pr(X_n\in A|X_0=x)\) where
\(A\) is a measurable set. \(\phi(A)=\int_A f_X(x)dx\). That is

\[\lim_{n\to\infty}||P^{n}(x,\cdot)-\phi(\cdot)||\to 0\]

where
\(||P^{n}(x,\cdot)-\phi(\cdot)||=\underset{A}{\sup}|P^{n}(x,A)-\phi(A)|\)

Let \(L^1(f_X)\) denote the set of functions
\(h: \mathbf{X}\to \mathbb{R}\) \(\int_X |h(x)|f_X(x)dx<\infty\) and
\(E_{f_{X}}h=\int_X h(x)f_X(x)dx\). For \(g\in L^1(f_X)\)

\[\bar g_n := \frac{1}{n}\sum_{i=0}^{n-1} g(X_i)\overset{a.s}{\to}E_{f_{X}}[g(x)]\]

\hypertarget{geometric-ergodicity}{%
\subsubsection{Geometric ergodicity}\label{geometric-ergodicity}}

For all \(x\in \mathbf{X}\) and all n=1,2,.., if exist a function
\(M: \mathbf{X}\to [0,\infty)\) and a constant \(0<\rho<1\)

\[||P^{n}(x,\cdot)-\phi(\cdot)||\le M(x) \rho^n\]

Then the Markov chain \(X\) is geometrically ergodic.

\begin{itemize}
\tightlist
\item
  Drift condition
\end{itemize}

Define \(V: \mathbf{X}\to[0,\infty)\) is unbounded off compact sets,
constants \(\lambda\in[0,1]\), and \(L\in\mathbb{R}\). If the
\emph{drift function} \(V\) exist and

\[E[V(X_{n+1}|X_n=x)]\le\lambda V(x)+L\]

Then the Markov chain \(X\) is geometrically ergodic.

\hypertarget{central-limit-theorems}{%
\subsubsection{Central Limit Theorems}\label{central-limit-theorems}}

Assume that \(X\) is geometrically ergodic and \(g\in L^2(f_X)\). Define
\(\sigma^2=E_{f_{X}}g^2-(E_{f_{X}}g)^2\) and
\(\kappa^2=\sigma^2+2\sum_{k=1}^\infty c_k< \infty\). As \(n\to\infty\)

\[\sqrt{n}(\bar g_n-E_{f_{X}}g)\overset{d}{\to}N(0,\kappa^2)\]

\hypertarget{minorization-regenration-and-alteernative-clt}{%
\subsubsection{Minorization regenration and alteernative
CLT}\label{minorization-regenration-and-alteernative-clt}}

\subsection{MCMC and EM algorithm}
\label{sec:latent}

We can extend data augmentation methods by some techniques both from
MCMC and EM (Expectation and Maximization) algorithm
\citep{mclachlanEMAlgorithmExtensions2008}. Define \(Y\) as the observed
data, which are sampled from a family of pdfs
\(\{p(y|\theta):\theta\in \Theta\}\). To find the feature (parameter) of
the true distribution, the target is the \(\theta\)-marginal posterior
density \(\pi(\theta|y)\) and corresponding expectations
\(E[\pi(\theta|y)]\). Suppose the \textbf{missing data},
\(z\in Z\subset \mathbb{R}^q\) can be identified. The \textbf{complete
data} posterior density as

\[\pi(\theta,z|y)=\frac{p(y,z|\theta)\pi(\theta)}{\int_\Theta\int_{Z}p(y,z|\theta)\pi(\theta)dzd\theta}=\frac{p(y,z|\theta)\pi(\theta)}{\int_{\Theta}p(y|\theta)\pi(\theta)d\theta}=\frac{p(y,z|\theta)\pi(\theta)}{c(y)}.\]

where \(p(y,z|\theta)\) represents the joint density, \(\pi(\theta)\)
denotes the prior density, \(c(y)\) is the marginal density of the data.
we can check that the \(\theta\)-marginal pdf is invariant:

\[\int_{Z}\pi(\theta,z|y)dz=\frac{\pi(\theta)}{c(y)}\int_{Z}p(y,z|\theta)dz=\frac{p(y|\theta)\pi(\theta)}{c(y)}=\pi(\theta|y).\]

This question can be solved by deterministic EM algorithm. Let
\(\theta^{i}\) denote the current guess to the mode of the observed
posterior \(p(\theta|y)\). \(p(\theta|z,y)\) is the augmented posterior.
\(p(z|\theta^{i},y)\) represents the conditional predictive distribution
of the latent data \(Z\), conditional on the current guess to the
posterior mode. In the E-step, we can find the \(Q\)-function, the
conditional expectation of the complete-data log likelihood function as

\[Q(\theta,\theta^{i})=\int_Z\log(p(\theta|z,y))p(z|\theta^{i},y)dz\]

In the M-step, \(Q\)-function is maximized with respect to \(\theta\) to
obtain \(\theta^{i+1}\) by Coordinate-Ascent Algorithm. Continue the
E-step and M-step until convergent. However, for the high-dimension
\(\theta\) the \(Q\)-function is hard to differentiate.

Once we can straightforwardly sample from two full posterior densities
\(\pi(z|\theta,y)\) and \(\pi(\theta|z,y)\), then the target density
\(\pi(\theta|y)\) can be simulated by MCMC
\citep{mengSeekingEfficientData1999}. EM algorithm works well with
exponential families when the log-concave property holds. But if
\(p(z|\theta,y)\) is not ``sufficiently smooth'', the iterates
\(\theta^{i}\) may converge to some local maxima, minima, or saddle
stationary point of \(p(\theta|y)\). In small samples, the posterior or
likelihood may not consistent with the normal approximation. Unlike the
EM algorithm, the DA methods base on the entire likelihood function or
posterior distribution. It can converge to the global maximizer and can
also improves the inference in both small and large sample situation
\citep{tannerToolsStatisticalInference1993}.

The EM algorithm converges at a linear
rate\citep{dempsterMaximumLikelihoodIncomplete1977}, with the rate
depending on the proportion of information about \(\theta\) in
\((\theta|Y)\) is observed. In contrast, MCMC can converge at a
geometric speed, More discussions of EM and DA can be found in
\citet{dykCrossFertilizingStrategiesBetter2010}.

\textbf{Example 4} The multivariate location-scale Student's \(t\)
density with known degree of freedom \(\nu>0\).

Assume the true distribution of response
\(Y_i\sim t_{\nu,\mu,\sigma^2}\),i=1,..,m.

\[p(y|\mu,\sigma^2)=\frac{\Gamma(\frac{\nu+1}2)}{\sqrt{\nu\sigma^2}\Gamma(\frac{\nu}2)}(1+\frac{(y-\mu)^2}{\nu\sigma^2})^{-\frac{\nu+1}2}\]

Let the prior density \(\pi(\mu,\sigma^2)\propto 1/\sigma^2\). The
target posterior density is

\[\begin{aligned}
\pi(\mu,\sigma^2|y)&\propto\pi(\mu,\sigma^2)\prod_{i=1}^m{p(y_i|\mu,\sigma^2)} \\
                   &\propto(\sigma^2)^{-\frac{m+2}2}\prod_{i=1}^m(1+\frac{(y_i-\mu)^2}{\nu\sigma^2})^{-\frac{\nu+1}2}
\end{aligned}\]

\citet{mengSeekingEfficientData1999} propose a solution by DA algorithm.
Let \(Z=\mathbb{R}^p_{+}\) and make a i.i.d. paired vectors
\((Y_{1:m},Z_{1:m})\), where
\((Y_{i}|Z_{i},\mu,\sigma^2)\sim N(\mu,\sigma^2/z_i)\),
\((Z_{i},\mu,\sigma^2) \sim Gamma(\nu/2,\nu/2)\). Then the joint density
is
\(p(y,z|\mu,\sigma^2)=\prod_{i=1}^mp(y_i|z_i,\mu,\sigma^2)p(z_i|\mu,\sigma^2)\)

Checking the \((\mu,\sigma^2)\)-marginal density confirms that the
condition holds.

\[\begin{aligned}\int_Y p(z,y|\mu,\sigma^2)dz=&\prod_{i=1}^m\int_{\mathbb{R}_{+}}\frac{\sqrt{z_i}}{\sqrt{2\pi\sigma^2}}\exp\left[-\frac{z_i}{2\sigma^2}(y_i-\mu)^2\right]\frac{(\frac{\nu}2)^{(\frac{\nu}2)}}{\Gamma(\frac{\nu}2)}z_i^{\frac{\nu}2-1}\exp\left[-\frac{\nu}{2}z_i\right]dz_i\\
=&\prod_{i=1}^m\frac{\Gamma(\frac{\nu+1}2)}{\sqrt{\nu\sigma^2}\Gamma(\frac{\nu}2)}(1+\frac{(y_i-\mu)^2}{\nu\sigma^2})^{-\frac{\nu+1}2}\end{aligned}\]

Then we can rearrange the joint pdf.

\[\begin{aligned}
p(\mu,\sigma^2,z|y)&\propto\pi(\mu,\sigma^2)p(z,y|\mu,\sigma^2)\\
                     &\propto\frac{1}{\sigma^2}\prod_{i=1}^m\frac{\sqrt{z_i}}{\sqrt{2\pi\sigma^2}}\exp\left[-\frac{z_i}{2\sigma^2}(y_i-\mu)^2\right]\frac{(\frac{\nu}2)^{(\frac{\nu}2)}}{\Gamma(\frac{\nu}2)}z_i^{\frac{\nu}2-1}\exp\left[-\frac{\nu}{2}z_i\right]\\
\end{aligned}\]

and get the first full conditional pdf

\[ \pi(z|\mu,\sigma^2,y)\propto \frac{(\frac{(y-\mu)^2}{2\sigma^2}+\frac{\nu}{2})^{(\frac{\nu+1}2)}}{\Gamma(\frac{\nu+1}2)}z^{\frac{\nu+1}2-1}\exp\left[-\left(\frac{(y-\mu)^2}{2\sigma^2}+\frac{\nu}{2}\right)z\right]\]

Define \(\hat\mu=\frac{1}{z_{.}}\sum_{j=1}^my_jz_j\) The second full
conditional pdf is

\[\begin{aligned}
\pi(\mu|\sigma^2,y,z)&\propto\exp\left[-\frac{\sum z_j}{2\sigma^2}\mu^2+\frac{\sum z_jy_j}{\sigma^2}\mu\right]\\
&\propto\exp\left[-\frac{z_.}{2\sigma^2}(\mu-\frac{\sum y_jz_j}{z_{.}})^2\right]\\
&\propto\frac{\sqrt{z_.}}{\sqrt{2\pi\sigma^2}}\exp\left[-\frac{z_.}{2\sigma^2}(\mu-\hat\mu)^2\right]
\end{aligned}\]

Define \(\hat\sigma^2=\frac{1}{z_{.}}\sum_{j=1}^mz_j(y_j-\hat\mu)^2\).
The third full conditional pdf can get by

\[\begin{aligned}
\pi(\sigma^2|y,z)&= \int_{\mathbb{R}}p(\mu,\sigma^2,z|y)d\mu\\
               &\propto \frac{(\frac{z_{.}\hat\sigma^2}2)^{\frac{m+1}2}}{\Gamma(\frac{m+1}2)} \sigma^{-m-1}\exp\left[-\frac{z_{.}\hat\sigma^2}{2\sigma^2}\right]
\end{aligned}\]

Thus, we can simulate \((\mu,\sigma^2)\) by three steps:

\begin{enumerate}
\def\labelenumi{\arabic{enumi}.}
\item
  Draw \((Z_i|\mu,\sigma^2,y)\) from
  \(Gamma(\frac{\nu+1}2,\frac{1}2(\frac{(y_i-\mu)^2}{\sigma^2}+\nu))\).
\item
  Draw \((\sigma^2|y,z)\) from
  \(IG(\frac{m+1}2,\frac{z_{.}\hat\sigma^2}2)\).
\item
  Draw \((\mu|\sigma^2,y,z)\) from
  \(N(\hat\mu,\frac{\sigma^2}{z_{.}})\).
\end{enumerate}

\hypertarget{probit-models}{%
\section{Probit Models}\label{probit-models}}

Probit models is one type of Generalized Linear Models. Suppose the
binary response \(Y_1,..,Y_n\sim Bern(p_i)\) is the observed data with
sample size \(n\). \(X_i^T=(X_{i1},..,X_{ip})\) known \(p\) covariates.
\(\beta_{k\times1}\) is the unknown vector of parameters we want to
estimate.

To construct a regression model, suppose a known CDF \(H(\cdot)\) as the
link function with the linear part \(\mathbf{x}_i^T\boldsymbol{\beta}\)
Then, let \(p_i=H(\mathbf{x}_i^T\boldsymbol{\beta})\).

Denote \(\pi(\boldsymbol{\beta})\) as the prior density. The target
conditional density is \[
\pi(\boldsymbol{\beta}|y) = \frac{\pi(\boldsymbol{\beta})}{C(y)}\prod_{i=1}^{k}H (\mathbf{x}_i^T\boldsymbol{\beta})^{y_i}(1-H(\mathbf{x}_i^T\boldsymbol{\beta}))^{1-y_i}
\]

where \(c(y)\) is free from \(\beta\)
\[C(y)=\int\pi(\boldsymbol{\beta})\prod_{i=1}^{k}H (\mathbf{x}_i^T\boldsymbol{\beta})^{y_i}(1-H(\mathbf{x}_i^T\boldsymbol{\beta}))^{1-y_i}d\boldsymbol{\beta}\]

When \(H(\cdot)\) represents the standard normal CDF, it obtains the
probit model. When \(H(\cdot)\) is logistic CDF, it is the logistic
model.

While the logistic models have a close form, probit models cannot
compute in closed form and are hard to integral.

Sampling from the posterior density \(\pi(\boldsymbol{\beta}|y)\) with
high dimension is also problematic. (why)

\citet{albertBayesianAnalysisBinary1993} propose a DA algorithm to
estimate the exact posterior distribution of \(\beta\). This
simulation-based approach introduces latent variables
\(Z_1,..Z_N\overset{iid}{\sim} N(\mathbf{x}_i^T\boldsymbol{\beta},1)\).
The relationship between observed data and missing data is
\(y_i=\begin{cases}1&\text{if } z_i>0\\0&\text{if }z_i\le0\end{cases}\).
The joint mass function is

\[\begin{aligned}
\pi(\boldsymbol{\beta,z|y}) &= \frac{\pi(\boldsymbol{\beta})}{C(y)}\prod_{i=1}^{N}[p(y_i|\beta)]f(z_i|\beta)\\
&\propto\pi(\boldsymbol{\beta})\prod_{i=1}^{N}\left[\Phi (\mathbf{x}_i^T\boldsymbol{\beta})^{y_i}(1-\Phi(\mathbf{x}_i^T\boldsymbol{\beta}))^{1-y_i}\right]\phi (z_{i}|\beta)\\
&=\pi(\boldsymbol{\beta})\prod_{i=1}^{N}\left[1_{z_i>0}1_{y_i=1}+1_{z_i\le0}1_{y_i=0}\right]\phi (z_{i}|\beta)\\
\end{aligned}\]

where \(\Phi\) is a standard normal CDF, \(\phi\) is a normal pdf with
means equal \(\mathbf{x}_i^T\boldsymbol{\beta}\) and variances equal
one.

Assign a flat prior to \(\beta\), we can get the first conditional
posterior density:

\[\pi(\boldsymbol{\beta|y,Z})\propto \prod_{i=1}^{N}\phi (Z_{i};\mathbf{x}_i^T\boldsymbol{\beta},1)\]

For a multivariate linear model
\(\mathbf{Z}=\mathbf{X}\boldsymbol{\beta+\varepsilon}\) where
\(\boldsymbol{\varepsilon}\sim N_N(0,\mathbf{I})\), the least squares
estimates have
\(\hat\beta=(\mathbf{X}^T\mathbf{X})^{-1}\mathbf{X}^T\mathbf{Z}\) and
\(\hat\sigma^2_\beta=(\mathbf{X}^T\mathbf{X})^{-1}\). Thus, we can draw
\(\boldsymbol{\beta|y,Z}\) from
\(N_k(\mathbf{X}^T\mathbf{X})^{-1}\mathbf{X}^T\mathbf{Z},(\mathbf{X}^T\mathbf{X})^{-1})\)

The second conditional posterior pdf is

\[Z_i|\boldsymbol{y,\beta}\sim N(\mathbf{x}_i^T\boldsymbol{\beta},1)\begin{cases}\text{truncated at the left by 0} & \text{if } y_i=1 \\
 \text{truncated at the right by 0} & \text{if } y_i=0 \end{cases}\]

\hypertarget{logistic-models}{%
\section{Logistic Models}\label{logistic-models}}

In logistic models, the link function
\(p_{i}=H(z_i)= \frac{e^{\mathbf{x}_i^T\boldsymbol{\beta}}}{1+e^{\mathbf{x}_i^T\boldsymbol{\beta}}}\).
The log odds of success \(H^{-1}(p_i)=\log(\frac{p_i}{1-p_i})\)

The joint pmf is

\[\begin{aligned}
\pi(\boldsymbol{\beta|y}) &= \frac{\pi(\boldsymbol{\beta})}{C(y)}\prod_{i=1}^{N}[p(y_i|\beta)]\\
&\propto\pi(\boldsymbol{\beta})\prod_{i=1}^{N}\left[(\frac{e^{\mathbf{x}_i^T\boldsymbol{\beta}}}{1+e^{\mathbf{x}_i^T\boldsymbol{\beta}}})^{y_i}(1-\frac{e^{\mathbf{x}_i^T\boldsymbol{\beta}}}{1+e^{\mathbf{x}_i^T\boldsymbol{\beta}}})^{1-y_i}\right]\\
&=\pi(\boldsymbol{\beta})\prod_{i=1}^{N}\left[\frac{e^{y_i\mathbf{x}_i^T\boldsymbol{\beta}}}{1+e^{\mathbf{x}_i^T\boldsymbol{\beta}}}\right]\\
\end{aligned}\]

\hypertarget{single-layer-da-methods}{%
\subsection{Single-layer DA methods}\label{single-layer-da-methods}}

\citet{polsonBayesianInferenceLogistic2013} propose a DA approach which
only need a single layer of latent variables, which involve the
Pólya--Gamma distribution. Denote
\(W_i \sim PG(1,\boldsymbol{|X'_{i,1:p}\beta|})\),i=1:n, then (no
absolute value in polson2013)

\[f(\omega|\beta)=\cosh{(\frac12\boldsymbol{x^T\beta})}\exp\left[-\frac12(\boldsymbol{x^T\beta})^2\omega\right]g(\omega)\]

where \(\cosh{c}=\frac12(e^c+e^{-c})=\frac{1+e^2c}{2e^{c}}\),
\(g(\omega)\) is free of \(\beta\) s.t.

\[g(\omega)=\sum_n^\infty(-1)^n\frac{2n+1}{\sqrt{2\pi\omega^3}}\exp[-\frac{(2n+1)^2}{8\omega}]\mathbb{I}_{(0,\infty)}(\omega)\]

Since \(y\) is observed data, \(\pi(\omega|\beta,y)=f(\omega|\beta)\).

Assign the prior of \(\boldsymbol{\beta}\) is
\(\sim N_p(b,\boldsymbol{B})\)

\[\begin{aligned} 
\pi(\boldsymbol{\beta,\omega|y})&= \frac{\pi(\boldsymbol{\beta})}{C(y)}\prod_{i=1}^{N}[p(y_i|\beta)]f(\omega_i|\beta)\\
&= \frac{\pi(\boldsymbol{\beta})}{C(y)}\prod_{i=1}^{N}\left[\frac{e^{y_i\mathbf{x}_i^T\boldsymbol{\beta}}}{1+e^{\mathbf{x}_i^T\boldsymbol{\beta}}}\right]\cosh{(\frac12\boldsymbol{x_i^T\beta})}\exp\left[-\frac12(\boldsymbol{x_i^T\beta})^2\omega_i\right]g(\omega_i)\\
&\propto\phi_p(\mathbf{b,B})\prod_{i=1}^{N}\frac{e^{y_i\mathbf{x}_i^T\boldsymbol{\beta}}}{1+e^{\mathbf{x}_i^T\boldsymbol{\beta}}}\cdot\frac{1+e^{\boldsymbol{x_i^T\beta}}}{2e^{\frac12\boldsymbol{x_i^T\beta}}}\exp\left[-\frac12(\boldsymbol{x_i^T\beta})^2\omega\right]\\
&=2^{-n}\phi_p(\mathbf{b,B})\prod_{i=1}^{N}\exp\left[(y_i-\frac12)\mathbf{x}_i^T\boldsymbol{\beta}-\frac12(\boldsymbol{x_i^T\beta})^2\omega_i\right]\\
\end{aligned}\]

When \(\omega_i\) is known,
\(\pi(\boldsymbol{\beta,\omega|y})=\pi(\boldsymbol{\beta|\omega,y})\).
Let \(\boldsymbol{\Omega}=diag_n(\mathbf{\omega_i})\);
\(\boldsymbol{\kappa}=\mathbf{y_{1:n}}-\frac12\) Then we get to
conditional pmfs of \(\pi(\boldsymbol{\beta|\omega,y})\)

\[\begin{aligned} 
\pi(\boldsymbol{\beta|\omega,y})&\propto\exp\left\{-\frac12\left[\boldsymbol{\beta'(X'\Omega X+B^{-1})\beta-2\beta'(X'\kappa+B^{-1}b)}\right]\right\}\\
&\propto\exp\left\{-\frac12\boldsymbol{V_\omega^{-1}[\beta'I\beta-2\beta'V_\omega(X'\kappa+B^{-1}b)]}\right\}\\
&\propto\exp\left\{-\frac12\boldsymbol{(\beta-m_\omega)'V_\omega^{-1}(\beta-m_\omega)}\right\}\\
\end{aligned}\] where
\(m_\omega=\underset{(p\cdot p)}{V_\omega}(\underset{(p\cdot n)}{X^T}(\mathbf{y_{1:n}}-\frac12)+B^{-1}b)\);
\(V_\omega=(X^T\Omega X+B^{-1})^{-1}\). Then we confirm that
\(\boldsymbol{\beta|\omega,y}\sim N_p(m_\omega,V_\omega)\)

To sample from the posterior distribution using the Pólya--Gamma method,
simply iterate two steps:

\[\begin{aligned}
(\omega_i|\boldsymbol{\beta})&\sim PG(n_i,x_i^T\boldsymbol{\beta})\\
(\boldsymbol{\beta}|y,\omega)&\sim N(m_\omega,V_\omega)
\end{aligned}\]

\hypertarget{other-bayesian-methods}{%
\subsection{Other Bayesian methods}\label{other-bayesian-methods}}

One method is similar with \citet{albertBayesianAnalysisBinary1993}. Let
\(y_i=\begin{cases}1&\text{if } z_i>0\\0&\text{if }z_i\le0\end{cases}\).
The \emph{latent utilities}
\(z_i=\mathbf{x}_i^T\boldsymbol{\beta}+\varepsilon_i\),
\(\varepsilon_i\sim Logistic(1)\)

The standard approach has been to add another layer of auxiliary
variables to handle the logistic error model on the latent-utility
scale. One strategy is to represent the logistic distribution as a
normal-scale mixture

\[(\epsilon_i|\phi_i)\sim N(0,\phi_i);\quad \phi_i=(2\lambda_i)^2; \lambda_i\sim KS(1)\quad\text{Kolmogorov–Smirnov distribution}\]

Alternatively, one may approximate the logistic error term as a discrete
mixture of normals.

\[(\epsilon_i|\phi_i)\sim N(0,\phi_i);\quad \phi_i=\sum_{k=1}^K\omega_k\delta_{\phi^{(k)}}\]

where \(\delta_{\phi}\) indicates a {\emph{Dirac measure}} at \(\phi\).
The weights \(\omega_k\) and the points \(\phi^{(k)}\) in the discrete
mixture are fixed for a given choice of \(k\) so that \emph{{the
Kullback--Leibler divergence}} from the true distribution of the random
utilities is minimized. Frühwirth-Schnatter and Frühwirth (2010) found
that the choice of \(K=10\) leads to a good approximation.

The discrete mixture of normals is an approximation, but it outperforms
the scale mixture of normals in terms of effective sampling rate, as it
is much faster.

\[\begin{tabular}{|l|l|l|}\hline
                 & {Albert and Chib (1993)} & Polson et al. (2013) \\\hline
{Gaussian}         & location mixture  & scale mixture \\
{Latent variables} & truncated normals & Polya-Gamma \\\hline
\end{tabular}\]

One may also arrive at the hierarchy above by manipulating the random
utility derivation of McFadden (1974)

The dRUM. One must use a table of different weights and variances
representing different normal mixtures, to approximate a finite
collection of type-III logistic distributions, and interpolate within
this table to approximate the entire family.

Another approximation: the use of a Student-t link function as a close
substitute for the logistic link. This also introduces a second layer of
latent variables, in that the Student-t error model for \(z_i\) is
represented as a scale mixture of normals.

\hypertarget{other-frequentist-methods}{%
\subsection{Other Frequentist methods}\label{other-frequentist-methods}}

\hypertarget{mle-methods}{%
\subsubsection{MLE methods}\label{mle-methods}}

\hypertarget{em-algoritm}{%
\subsubsection{EM Algoritm}\label{em-algoritm}}

\hypertarget{implementation}{%
\subsection{Implementation}\label{implementation}}

\hypertarget{summary}{%
\subsection{Summary}\label{summary}}

\emph{Our data-augmentation scheme differs from each of these approaches
in several ways.}

\emph{it does not appeal directly to the random-utility interpretation
of the logit model. Instead, it represents the logistic CDF as a mixture
with respect to an infinite convolution of gammas.}

\emph{the method is exact, in the sense of making draws from the correct
joint posterior distribution, rather than an approximation to the
posterior that arises out of an approximation to the link function.}

\emph{like the Albert and Chib (1993) method, it requires only a single
layer of latent variables.}

\emph{In binary logit models, the Pólya--Gamma is more efficient than
all previously proposed data-augmentation schemes.}

\emph{the Pólya--Gamma method always had a higher effective sample size
than the two default Metropolis samplers we tried. }

\emph{the Pólya--Gamma method truly shines when the model has a complex
prior structure.}

\bibliographystyle{agsm}
\bibliography{stat501.bib}

\end{document}
