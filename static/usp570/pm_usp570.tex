\documentclass[12pt,]{article}
\usepackage{lmodern}
\usepackage{amssymb,amsmath}
\usepackage{ifxetex,ifluatex}
\usepackage{fixltx2e} % provides \textsubscript
\ifnum 0\ifxetex 1\fi\ifluatex 1\fi=0 % if pdftex
  \usepackage[T1]{fontenc}
  \usepackage[utf8]{inputenc}
\else % if luatex or xelatex
  \ifxetex
    \usepackage{mathspec}
  \else
    \usepackage{fontspec}
  \fi
  \defaultfontfeatures{Ligatures=TeX,Scale=MatchLowercase}
    \setmainfont[]{Times New Roman}
\fi
% use upquote if available, for straight quotes in verbatim environments
\IfFileExists{upquote.sty}{\usepackage{upquote}}{}
% use microtype if available
\IfFileExists{microtype.sty}{%
\usepackage{microtype}
\UseMicrotypeSet[protrusion]{basicmath} % disable protrusion for tt fonts
}{}
\usepackage[margin=1in]{geometry}
\usepackage{hyperref}
\hypersetup{unicode=true,
            pdftitle={Policy Memo},
            pdfborder={0 0 0},
            breaklinks=true}
\urlstyle{same}  % don't use monospace font for urls
\usepackage{graphicx,grffile}
\makeatletter
\def\maxwidth{\ifdim\Gin@nat@width>\linewidth\linewidth\else\Gin@nat@width\fi}
\def\maxheight{\ifdim\Gin@nat@height>\textheight\textheight\else\Gin@nat@height\fi}
\makeatother
% Scale images if necessary, so that they will not overflow the page
% margins by default, and it is still possible to overwrite the defaults
% using explicit options in \includegraphics[width, height, ...]{}
\setkeys{Gin}{width=\maxwidth,height=\maxheight,keepaspectratio}
\IfFileExists{parskip.sty}{%
\usepackage{parskip}
}{% else
\setlength{\parindent}{0pt}
\setlength{\parskip}{6pt plus 2pt minus 1pt}
}
\setlength{\emergencystretch}{3em}  % prevent overfull lines
\providecommand{\tightlist}{%
  \setlength{\itemsep}{0pt}\setlength{\parskip}{0pt}}
\setcounter{secnumdepth}{0}
% Redefines (sub)paragraphs to behave more like sections
\ifx\paragraph\undefined\else
\let\oldparagraph\paragraph
\renewcommand{\paragraph}[1]{\oldparagraph{#1}\mbox{}}
\fi
\ifx\subparagraph\undefined\else
\let\oldsubparagraph\subparagraph
\renewcommand{\subparagraph}[1]{\oldsubparagraph{#1}\mbox{}}
\fi

%%% Use protect on footnotes to avoid problems with footnotes in titles
\let\rmarkdownfootnote\footnote%
\def\footnote{\protect\rmarkdownfootnote}

%%% Change title format to be more compact
\usepackage{titling}

% Create subtitle command for use in maketitle
\providecommand{\subtitle}[1]{
  \posttitle{
    \begin{center}\large#1\end{center}
    }
}

\setlength{\droptitle}{-2em}

  \title{Policy Memo}
    \pretitle{\vspace{\droptitle}\centering\huge}
  \posttitle{\par}
  \subtitle{Autonomous vehicles and land use}
  \author{}
    \preauthor{}\postauthor{}
      \predate{\centering\large\emph}
  \postdate{\par}
    \date{June 11}


\begin{document}
\maketitle

To: Scott Haggerty, chair of governing Commission of Bay Area
Metropolitan Transportation Commission

From: Shen Qu, Policy Advisor

Date: 6/11/2019

RE: How Bay Area should be planning for autonomous vehicles?

\hypertarget{summary}{%
\section{Summary}\label{summary}}

This memo is one of the series of policy analysis about autonomous
vehicles for Bay Area. It trys to answer how the autonomous vehicles
will affect urban land use, and What the MPO and cities should be
planning to seize this opportunity and address the challenges.

\hypertarget{background-the-current-and-projected-status-of-avs.}{%
\section{Background: the current and projected status of
AVs.}\label{background-the-current-and-projected-status-of-avs.}}

\emph{Technology}: The history of exploring autonomous vehicles can date
back to late 1950s (Milakis 2019). Since DARPA ran the Grand Challenge
in 2004, the autonomous vehicle technologies entered a ``critical
juncture.'' (Docherty, Marsden, and Anable 2018) In the fields of
automation control systems, some critical hardwares like processors and
sensors are having the ability of undertaking more complex tasks. The
improving softwares and algorithm are becoming more fledged. According
the third version of definition and taxonomy by SAE (2018), the existing
technologies are achiving from the Level 3 - Conditional Driving
Automation to Level 4 - High Driving Automation.\footnote{SAE defines
  the concept of AV as ADS-DV (ADS-Dedicated Vehicle), ``A vehicle
  designed to be operated exclusively by a level 4 or level 5 ADS for
  all trips within its given Operational Design Domain (ODD)
  limitations.'' ADS means ``The hardware and software that are
  collectively capable of performing the entire Dynamic Driving Task
  (DDT) on a sustained basis, regardless of whether it is limited to a
  specific ODD; this term is used specifically to describe a level 3, 4,
  or 5 driving automation system.''}

\emph{Industry}: Since 2009, Google had conducted a series of tests for
AVs over 10 million miles on real-world roads in California, Texas, and
other states. Waymo, a company founded by Google, hold the only testing
permit for driverless testing by California DMV and committed to
providing a ride-hailing services in Arizona in 2018. In 2019, Waymo
announced their Level 4 AV will be assembled in Detroit. Almost all big
automakers such as Ford, General Motors, Volkswagen, etc., are investing
heavily in this field.(Crute et al. 2018) The leading transportation
network companys (TNC) like Uber and Lyft are making up a term of
Mobility-as-a-service (Maas) to change current travel modes by AVs.

\emph{Academia}: Many scholars start working on the reaserch of AVs with
lots of energy. Gandia et al. (2019)`s reserch found 10580 published
papers in this field from 1945 to 2018. Since 2012, the number of
articles have an exponential growth with 39\% growth rate while 8-9\%
average growth rate in science. Although a large amount of the research
are from the perspective of systems control, computer science, robotics,
engineering, there are more and more articls that start to focus on the
AVs' impact on transportation and Land use.

\emph{Governavce}: In 2011, the Nevada Department of Motor Vehicles
issued a first license to google's experimental AVs. Currently, ``33
states have passed legislation related to AV.'' ``15 states enacted 18
AV related bills.'' (NCSL 2019) From 2016 to 2018 the U.S. Department of
Transportation (U.S.DOT) and the National Highway and Transportation
Safety Administration (NHTSA) published three federal guidences for
Automated Driving Systems (ADS is the definition by SAE). The guidances
advocate industry, state and local government to support the AVs'
development. (NHTSA 2019) California is playing a leading role in this
field. Until January 2019, California DMV has issued AV Testing Permits
(with a driver) to 62 companies on public roadways.

Unveiling the future: Muller (2017) introduces the four stages in the
spatial evolution of the American metropolitan. Form Walking-Horsecar
Ear to Electric Streetcar Era, Recreational Auto Era, and Freeway Era.
The four-stage model shows that each ``break through in movement
technology'' had reshaped the previous dominated urban form and launched
a new era with a ``distinctive spatial structure.'' There are some
indications that a new era might be dawning. We try to answer how AVs
could influence demand for transportation and land use. This memo will
focus on the crutial response for land use. The topic of safety,
liability, and other issues will discussed in other memos.

\hypertarget{the-framworks-of-changes}{%
\section{The Framworks of changes}\label{the-framworks-of-changes}}

While industry, scholars and governments realized the high impact of AV,
some research frameworks also imply it is a highly uncertainly evolution
rising up some complex issues such as coupling, resonance, or agitation.
Milakis, Van Arem, and Van Wee (2017) arrange many substantial
implications of autonomous vehicles by a structure of {[}\emph{ripple
effect}{]}, whcih reflected a sequantially spreading process. Land use
is placed in the second-order that is affected by the factors in the
first-order including travel cost, travel time, vehicle use, capacity,
trvael modes, and etc. The flaw of this structure is that the ripple
effects model emphasizes the diffusion characteristic of the AV
technology and cannot describe the feedback effects. The changes of real
estate and land use will influence the travel behavior and traffic in
the first-order too.

The \emph{Diamond of Assembly} (Levinson and Krizek 2018 Chapter.12) and
the \emph{feedback cylce} (Wegener and Fürst 2004; adaped by
Soteropoulos, Berger, and Ciari 2019) are two helpful complements. These
figures can present the relationship between transportation and land use
in a more clear manner. AV technology as a exogenous variable will
influence travel behaviors in the first ripple, and then is reflected in
the change of accessiblity. The dynamic of accesessibility will interact
with land use and forms a new cycle.

The history of `the four stages' (Muller 2017) tells us, if AVs being a
breakthrough force, many previous models, methods, and arguments may be
different or even fail. Under the two kinds of the framework above, we
will review the classical theories about transportation and land use,
following the sequence of the first order, second order, and feedback
cycle.

\hypertarget{the-first-order}{%
\section{The first order}\label{the-first-order}}

\emph{The utility maximization problem}: As the core of consumer theory,
the supply and demand model explaines the relationship between the price
(travel cost) and the quantity (VMT). When price changed or the curve
shifted, the market-clearing equilibrium point will reach a new one. In
estimating the impact of AV on travel consumption, most of current
research agree that AV will provide more options for travelers with less
travel time and larger capacity {[}{]}. These means the lower price and
higher service qualities, which will induce more and longer trips,
produce more overall VMT {[}{]}. At the same time, other research noice
that more available services by AV may encourage low-occupied vehile and
reduce transit use. {[}{]} The tradeoff between higher effecient and
higher demand makes the trends of congestion and emission being
uncertaint. Overall, these are incremental changes in quantities.

However, as a breakthrough technology, the impact of AVs may not follow
some divergent curves. For example, the \emph{prospect theory} of value
and gains shows a logistic curve. (Levinson and Krizek 2018, 6) When
coming to the relationship of bus ridership to wait time, the curve
shows the initial point is not stable. (Levinson and Krizek 2018, 83) A
slightly change will let it slipe to two stabel equilibria. But the
equilibrious points represent two direction, a vicious circle (low
ridership and high wait time) or a virtuous circle (high ridership and
low wait time).

To explain this phenomenon, we need understand the components of travel
cost and current options on travel modes. The opportunity cost and
transaction cost are the significant part of travel cost. Being late for
work, a meeting, or a flight means much higher opportunity loss than
travel cost, whcih can explain why people hat uncertaint and wait time.
Similarly, hailing a taxi for each trip or planning carpool everyday
means the transaction cost are unacceptable expensive. Meanwhile, the
available travel modes options are limited. Once people choosing
transit, their trips often lost the reliability and flexibility. They
also have few options on the transit quality, such as speed, headway,
and routes. Thus, people choose to own a private car to control the high
opportunity cost and transaction cost. Now we know why the initial point
is not stable.

The practices of mobility-as-a-services (Maas) proved by TNC introduced
a potential solution. For the connected AV's `perfect information' and
automation, Mass could largely reduce the opportunity cost and
transaction cost. Moreover, It provdes a series of options by monetized
the cost of reliability and flexibility. A hurried passenger may will
pay \$100 for arriving the airport as soon as possible. A worker may
will get up one hour earlier for a \$10 off-peak pass per month. Many
research show that the ridersharing level is a critical variable for AV
application. The preseting value of ridershare could drastically change
the simulating results of the models. {[}{]}

As shown in Milakis, Van Arem, and Van Wee (2017)'s ripple effect, these
changes triggered by AVs are not about land use. But they are the direct
forces pushing the land use changes forward. And the changes may like a
suddenly switching lanes rather than a gradual process. We should
realize the impacts of AVs may keep accelerating until all the energy
released and change the prvious equations on parameters and even
distributions.

\hypertarget{the-second-order}{%
\section{The second order}\label{the-second-order}}

\emph{The bid-rent theroy}: this theory oriented form von {[}Thünen{]}`s
model, which derived from the utility maximization through introducing
the spatial variables. The essential reason is that the travel cost is
positively correlated with distance. All the activities want to minimize
the cost and compete for the land close to city center. From the
'concentric zone model' by {[}Burgess{]} to {[}Alonso{]}'s land market
model, and to the gravity model for measuring accessibility, this
economic theory is the basis of many land use models. The relevant
research conclude that urban sprawl might be an output of AV
application. {[}{]} In the same way, the research about parking say that
the CBD could be more dense for less demand of parking lot. However,
these inferences build on the assumptions that the functions remain the
same. the incremental changes happen on values and parameters.

The theory of \emph{network society} by Castells (2011) and other theory
of the information society imply another possible perspect of land use
pattern in the AVs era. If the travel demands and supplies can match in
realtime and generate as many as possible options for customers, the
`space of places' also could become the `space of flows.' The
traditional forms, including cores, clusters, and corridors will be
disintegrated. In the long run, The network may control the travel and
assign people to a place. People only choose their activities and
surrounding environment but don't care about the actual locations.

\hypertarget{the-feedback-cylce}{%
\section{The feedback cylce}\label{the-feedback-cylce}}

In this section, we try to explore what changes may happen on land use
feedback cycle brought by the AVs technology. selfreinforcement. The
changes triggered by AV may like a process of convolution and iteration,
Is AV will leverage the vicious or virtuous circle?

The concept of \emph{Public good} also helps to understand this topic.
Transprotation infrastructure is the semi-public good. Except toll road,
this system is non-excludable. But the congested sections in the peak
hour are rivalrous. Fuel taxs partly reflect the amount of use of road,
but can not adjust the spatial distribution. Even the congestion
charging like London, is still a binary intervention (Yes or no). Some
research conclude that the success of pricing policy depend on political
support. {[}{]} But the research don't realize that the current charging
plan cannot effectively and accurately reflect the use of transprotation
infrastructure. From the perspective of equalty, charging an
unmeasurable service is unfair. From the perspective of market, an
unmeasurable trade is inevitablelly inefficient. Pricing by
demand-supply mechanism becomes infeasible. Although many funding tools
try to combine the transportation investiment and land value return
together to incentive virtuous circle, there always some drawbacks.
{[}{]} The feedback cylce becomes a decay process for free rider
problem, or for delay of demand response.

Here we have to think about the \emph{Tiebout model}. As a positive
political theory model, (Tiebout 1956) contributes a non-political
solution to optimal public goods provision which known as ``foot
voting''. Through competition between communities, the mobile residents
``vote with their feet''. The primary assumptions Tiebout model relied
on are that conusmers can freely choose where they live. It assumes that
there are engough commuinties available and commuting cost is
negligible. there are not externalities or spillover of public goods
across towns. Many scholars critique Tiebout model for its unpractical
assumptions. For the same reasons, Tiebout model is more success in
suburban areas. The roads in suburban communities like a club goods.
There is no free rider problem. All the cost and externalities
internalized in property tax.

Therefore, the theory of local governments competetion and beneficiary
pays principle can inspire us to conceive the new feedback cycle. Maas
suppored by connected AV allows us to measure the road use for each
trip. Moreover, Maas provide the oppotuinity of internalizing all the
positive/negative externalities and redefine the boundary of public
good. Under this scenario, transportation become a excludable and
rivalrous goods. The cities or communities compete by investing and
improving infrastucture; Transportation suppliers bid for road license
and provide as many as possible options for consumers; Consumers choose
the travel services based on their ability and willingness to pay money
or change their itineraries. As Levinson and Krizek (2018) say, ``prices
create choices, and choices are fair.'' (p.248) Entering the AV era,
Tiebout model might apply to whole urban area. As simulated by social
ecological models, the social segregation may be more significant.

Behavior and land use (Soteropoulos, Berger, and Ciari 2019)

(Hawkins and Nurul Habib 2019)

long-term effects (Milakis 2019)

(Fagnant and Kockelman 2015)

\hypertarget{the-policy-options}{%
\section{The policy options}\label{the-policy-options}}

This solution of Maas also provides a valuable space for policy-makers
to guide the transportation system in the direction we want.

Above analysis are inference base on exsiting theories and research.

It's hard to conduct empirical studies of Land-use Tansport interaction
under the AV technology.

The stated preference, reveled preference, simulation by models

\begin{itemize}
\tightlist
\item
  The response for these incremental changes:
\end{itemize}

The ripple effects

``Mobility-as-a-Service may cause a decline in car ownership. If average
vehicle occupancy for on-road time decreases, total VMT will increase.''
(Taiebat et al. 2018) Time cost may decrease (Singleton 2019) cut off
labor cost, the change of parking, affected by travel demand and
behavior

the change on road capacities, parking lots, curve space.

\begin{itemize}
\tightlist
\item
  The response for these shifting changes:
\end{itemize}

Heavy property and light property.

feedback cycles produce

housing, urban design,

\begin{itemize}
\tightlist
\item
  The recommended initiative changes:
\end{itemize}

Research: Identifies the benefits and costs of these possible outcomes.

round-the-clock services.

full ridesharing by realtime matching

the impact on land use,

The long-term influences include the reconstructure of urban forms and
spatial distributions.

use cost and transaction costs - full match

deals fail

adjusting, adapting, guiding

(Legacy et al. 2019)

Presents policy and planning options for mitigating or otherwise
addressing the possible land use effects.

designating pilot area

Discusses how the MPO and cities may need alter the tools and analyses
they use to consider AVs.

Zoning, Division, and partion, not uniform

\hypertarget{conclusion}{%
\section{Conclusion}\label{conclusion}}

overestimated and under estimate

from link to node

CA should play a leading role. responsibility

\hypertarget{notes}{%
\section{Notes}\label{notes}}

\hypertarget{references}{%
\section*{References}\label{references}}
\addcontentsline{toc}{section}{References}

\hypertarget{refs}{}
\leavevmode\hypertarget{ref-castells2011rise}{}%
Castells, Manuel. 2011. \emph{The Rise of the Network Society}. Vol. 12.
John wiley \& sons.

\leavevmode\hypertarget{ref-APA2018autonomous}{}%
Crute, Jeremy, William Riggs, Timothy Stewart Chapin, and Lindsay
Stevens. 2018. ``Planning for Autonomous Mobility.'' PAS Report 592.
American Planning Association.

\leavevmode\hypertarget{ref-docherty2018governance}{}%
Docherty, Iain, Greg Marsden, and Jillian Anable. 2018. ``The Governance
of Smart Mobility.'' \emph{Transportation Research Part A: Policy and
Practice} 115. Elsevier: 114--25.
\url{https://doi.org/10.1016/j.tra.2017.09.012}.

\leavevmode\hypertarget{ref-fagnant2015preparing}{}%
Fagnant, Daniel J, and Kara Kockelman. 2015. ``Preparing a Nation for
Autonomous Vehicles: Opportunities, Barriers and Policy
Recommendations.'' \emph{Transportation Research Part A: Policy and
Practice} 77. Elsevier: 167--81.

\leavevmode\hypertarget{ref-gandia2019autonomous}{}%
Gandia, Rodrigo Marçal, Fabio Antonialli, Bruna Habib Cavazza, Arthur
Miranda Neto, Danilo Alves de Lima, Joel Yutaka Sugano, Isabelle
Nicolai, and Andre Luiz Zambalde. 2019. ``Autonomous Vehicles:
Scientometric and Bibliometric Review.'' \emph{Transport Reviews} 39
(1). Taylor \& Francis: 9--28.
\url{https://doi.org/10.1080/01441647.2018.1518937}.

\leavevmode\hypertarget{ref-hawkins2019integrated}{}%
Hawkins, Jason, and Khandker Nurul Habib. 2019. ``Integrated Models of
Land Use and Transportation for the Autonomous Vehicle Revolution.''
\emph{Transport Reviews} 39 (1). Taylor \& Francis: 66--83.
\url{https://doi.org/10.1080/01441647.2018.1449033}.

\leavevmode\hypertarget{ref-legacy2019planning}{}%
Legacy, Crystal, David Ashmore, Jan Scheurer, John Stone, and Carey
Curtis. 2019. ``Planning the Driverless City.'' \emph{Transport Reviews}
39 (1). Taylor \& Francis: 84--102.
\url{https://doi.org/10.1080/01441647.2018.1466835}.

\leavevmode\hypertarget{ref-levinson2018metropolitan}{}%
Levinson, David M, and Kevin J Krizek. 2018. \emph{Metropolitan Land Use
and Transport: Planning for Place and Plexus}. Routledge.
\url{https://doi.org/10.4324/9781315684482}.

\leavevmode\hypertarget{ref-milakis2019long}{}%
Milakis, Dimitris. 2019. ``Long-Term Implications of Automated Vehicles:
An Introduction.'' Taylor \& Francis.
\url{https://doi.org/10.1080/01441647.2019.1545286}.

\leavevmode\hypertarget{ref-milakis2017policy}{}%
Milakis, Dimitris, Bart Van Arem, and Bert Van Wee. 2017. ``Policy and
Society Related Implications of Automated Driving: A Review of
Literature and Directions for Future Research.'' \emph{Journal of
Intelligent Transportation Systems} 21 (4). Taylor \& Francis: 324--48.
\url{https://doi.org/10.1080/15472450.2017.1291351}.

\leavevmode\hypertarget{ref-Muller2017transportation}{}%
Muller, Peter O. 2017. ``Transportation and Urban Form.'' In \emph{The
Geography of Urban Transportation, Fourth Edition}, edited by G.
Giuliano and S. Hanson, 57--85. Guilford Publications.
\url{https://books.google.com/books?id=J3GnDQAAQBAJ}.

\leavevmode\hypertarget{ref-NCSL2019AV}{}%
NCSL. 2019. ``Autonomous Vehicles \textbar{} Self-Driving Vehicles
Enacted Legislation.'' National Conference of State Legislatures. 2019.
\url{http://www.ncsl.org/research/transportation/autonomous-vehicles-self-driving-vehicles-enacted-legislation.aspx}.

\leavevmode\hypertarget{ref-NHTSA2019ADS}{}%
NHTSA. 2019. ``Automated Driving Systems.'' National Highway;
Transportation Safety Administration. 2019.
\url{https://www.nhtsa.gov/vehicle-manufacturers/automated-driving-systems}.

\leavevmode\hypertarget{ref-sae2018taxonomy}{}%
SAE. 2018. ``Taxonomy and Definitions for Terms Related to Driving
Automation Systems for on-Road Motor Vehicles.'' J3016. SAE
International.

\leavevmode\hypertarget{ref-singleton2019discussing}{}%
Singleton, Patrick A. 2019. ``Discussing the `Positive Utilities' of
Autonomous Vehicles: Will Travellers Really Use Their Time
Productively?'' \emph{Transport Reviews} 39 (1). Taylor \& Francis:
50--65. \url{https://doi.org/10.1080/01441647.2018.1470584}.

\leavevmode\hypertarget{ref-soteropoulos2019impacts}{}%
Soteropoulos, Aggelos, Martin Berger, and Francesco Ciari. 2019.
``Impacts of Automated Vehicles on Travel Behaviour and Land Use: An
International Review of Modelling Studies.'' \emph{Transport Reviews} 39
(1). Taylor \& Francis: 29--49.
\url{https://doi.org/10.1080/01441647.2018.1523253}.

\leavevmode\hypertarget{ref-taiebat2018review}{}%
Taiebat, Morteza, Austin L Brown, Hannah R Safford, Shen Qu, and Ming
Xu. 2018. ``A Review on Energy, Environmental, and Sustainability
Implications of Connected and Automated Vehicles.'' \emph{Environmental
Science \& Technology} 52 (20). ACS Publications: 11449--65.
\url{https://doi.org/10.1021/acs.est.8b00127}.

\leavevmode\hypertarget{ref-tiebout1956pure}{}%
Tiebout, Charles M. 1956. ``A Pure Theory of Local Expenditures.''
\emph{Journal of Political Economy} 64 (5). The University Press of
Chicago: 416--24. \url{https://doi.org/10.1086/257839}.

\leavevmode\hypertarget{ref-wegener2004land}{}%
Wegener, Michael, and Franz Fürst. 2004. ``Land-Use Transport
Interaction: State of the Art.'' \emph{Available at SSRN 1434678}.
\url{https://doi.org/10.2139/ssrn.1434678}.


\end{document}
