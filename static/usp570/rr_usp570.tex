\documentclass[12pt,]{article}
\usepackage{lmodern}
\usepackage{amssymb,amsmath}
\usepackage{ifxetex,ifluatex}
\usepackage{fixltx2e} % provides \textsubscript
\ifnum 0\ifxetex 1\fi\ifluatex 1\fi=0 % if pdftex
  \usepackage[T1]{fontenc}
  \usepackage[utf8]{inputenc}
\else % if luatex or xelatex
  \ifxetex
    \usepackage{mathspec}
  \else
    \usepackage{fontspec}
  \fi
  \defaultfontfeatures{Ligatures=TeX,Scale=MatchLowercase}
    \setmainfont[]{Times New Roman}
\fi
% use upquote if available, for straight quotes in verbatim environments
\IfFileExists{upquote.sty}{\usepackage{upquote}}{}
% use microtype if available
\IfFileExists{microtype.sty}{%
\usepackage{microtype}
\UseMicrotypeSet[protrusion]{basicmath} % disable protrusion for tt fonts
}{}
\usepackage[margin=1in]{geometry}
\usepackage{hyperref}
\hypersetup{unicode=true,
            pdftitle={Reading reflections},
            pdfauthor={Shen Qu},
            pdfborder={0 0 0},
            breaklinks=true}
\urlstyle{same}  % don't use monospace font for urls
\usepackage{graphicx,grffile}
\makeatletter
\def\maxwidth{\ifdim\Gin@nat@width>\linewidth\linewidth\else\Gin@nat@width\fi}
\def\maxheight{\ifdim\Gin@nat@height>\textheight\textheight\else\Gin@nat@height\fi}
\makeatother
% Scale images if necessary, so that they will not overflow the page
% margins by default, and it is still possible to overwrite the defaults
% using explicit options in \includegraphics[width, height, ...]{}
\setkeys{Gin}{width=\maxwidth,height=\maxheight,keepaspectratio}
\IfFileExists{parskip.sty}{%
\usepackage{parskip}
}{% else
\setlength{\parindent}{0pt}
\setlength{\parskip}{6pt plus 2pt minus 1pt}
}
\setlength{\emergencystretch}{3em}  % prevent overfull lines
\providecommand{\tightlist}{%
  \setlength{\itemsep}{0pt}\setlength{\parskip}{0pt}}
\setcounter{secnumdepth}{0}
% Redefines (sub)paragraphs to behave more like sections
\ifx\paragraph\undefined\else
\let\oldparagraph\paragraph
\renewcommand{\paragraph}[1]{\oldparagraph{#1}\mbox{}}
\fi
\ifx\subparagraph\undefined\else
\let\oldsubparagraph\subparagraph
\renewcommand{\subparagraph}[1]{\oldsubparagraph{#1}\mbox{}}
\fi

%%% Use protect on footnotes to avoid problems with footnotes in titles
\let\rmarkdownfootnote\footnote%
\def\footnote{\protect\rmarkdownfootnote}

%%% Change title format to be more compact
\usepackage{titling}

% Create subtitle command for use in maketitle
\providecommand{\subtitle}[1]{
  \posttitle{
    \begin{center}\large#1\end{center}
    }
}

\setlength{\droptitle}{-2em}

  \title{Reading reflections}
    \pretitle{\vspace{\droptitle}\centering\huge}
  \posttitle{\par}
  \subtitle{USP 570}
  \author{Shen Qu}
    \preauthor{\centering\large\emph}
  \postauthor{\par}
      \predate{\centering\large\emph}
  \postdate{\par}
    \date{Week 6}


\begin{document}
\maketitle

\begin{itemize}
\tightlist
\item
  `Diamond of Evaluation'
\end{itemize}

Levinson and Krizek (2018 Chapter.10) introduced five criteria for
evaluating transportation and land use planning, which are called the
Diamond of Evaluation comprising the five ``Es.'' Efficiency and equity
are classic perspective of analysis. Environmental impacts is also a
widely accepted perspective. The measures of experience involves some
comprehensive factors. The final criterion, expdiency, is more like a
mechanism for decision-making and weighing the options. The five points
of view reflcet the complexcity of urban transportation and land use
system, and then result differing claims or proposed solutions. The
author also mentioned another evaluation pradiams, which including four
types of architechtures (functional, physical, techincal, and dynamic
operational) and four attibutes (robustness, adaptability, flexibility,
and schalability).

\begin{itemize}
\tightlist
\item
  Replacing LOS with VMT
\end{itemize}

The Senate Bill 743 in California is a meaningful change of
environmental impact assessment. It replaced previous measure of auto
congestion, level of service (LOS), with vehicle miles traveled (VMT).
From the perspective of efficiency, policy-makers and public realized
relying on LOS is not the solution but reinforces the traffic problems.
Road supplyment will never catch up the growing of demand, enforce the
auto dependency, and ``trap cities in an endless cycle of road-widening
projects.'' The evaluation of efficientcy also should focus on people's
needs and activities, consider the regional effects, long-term goals of
transportation and land use as a whole, not limit in road network
itself. The VMT metric relies on fewer assumptions and is cheaper.
Moreover, VMT can better reflect the outcomes in regional scale, can
capture a variety of widely recognized negative social, environmental,
and land-use impacts.

\begin{itemize}
\tightlist
\item
  Disscution: Diamond or Onion?
\end{itemize}

Thinking of the relationship between the five `Es', I find an oninon
structure may better discribe the relationship among them. The five `Es'
don't play equal roles. Efficiency is the the primary power of running,
the core value of evaluation. Other outlayers such as equtiy,
evirnonment, and expediency are adjustment tools to make the evaluation
more complete. We know a single perspective of efficiency is imperfect.
However, a single perspective of environment or expediecy doesn't work.
In the process of decision-making, efficiency is the step one. We
firstly need prove the benefits and then consider the options for
improving equity and other issues. If puting equity or envrionment on
the central place and treating efficiency as an ancillary position, the
whole system will slow down even break up. In the case of SB 743 in
California, GHG emissions reduction, human health and economic growth
are the primary reason, which all belone the generalized range of
efficiency. Less VMT responses the modal equity but doesn't help other
equity issue like ``the last ones in are the ones who pay.''

It is necessary to clarify that using the partial or short-term
efficeincy to evalue a project is a misunderstand. The case of SB 743
shows that we realize the traffic efficiency doesn't represent the
social efficiency and even hurt the overall and long-term efficiency.
This is the primary reason of change. Coming back to the five attributes
of good measures of effectiveness, VMT is clear, calculable, and
comparable. But it is more like a indicator that less is better. It
can't tell us the whole gain and loss. The utility by economists is
still the best measure for the transport--land use system. Some economic
concepts and methods, such as the value of a statistical life (VSL) and
the cost of climate change, try to integrate the different perspectives,
and provide an uniform metric as wide as possible. Expediency is that
VMT is the best criteria among the feasible tools in determining the
transportation impacts of projects in local level.

\hypertarget{references}{%
\section*{References}\label{references}}
\addcontentsline{toc}{section}{References}

\hypertarget{refs}{}
\leavevmode\hypertarget{ref-levinson2018metropolitan}{}%
Levinson, David M, and Kevin J Krizek. 2018. \emph{Metropolitan Land Use
and Transport: Planning for Place and Plexus}. Routledge.
\url{https://doi.org/10.4324/9781315684482}.


\end{document}
