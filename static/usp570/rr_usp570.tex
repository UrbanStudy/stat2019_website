\documentclass[12pt,]{article}
\usepackage{lmodern}
\usepackage{amssymb,amsmath}
\usepackage{ifxetex,ifluatex}
\usepackage{fixltx2e} % provides \textsubscript
\ifnum 0\ifxetex 1\fi\ifluatex 1\fi=0 % if pdftex
  \usepackage[T1]{fontenc}
  \usepackage[utf8]{inputenc}
\else % if luatex or xelatex
  \ifxetex
    \usepackage{mathspec}
  \else
    \usepackage{fontspec}
  \fi
  \defaultfontfeatures{Ligatures=TeX,Scale=MatchLowercase}
    \setmainfont[]{Times New Roman}
\fi
% use upquote if available, for straight quotes in verbatim environments
\IfFileExists{upquote.sty}{\usepackage{upquote}}{}
% use microtype if available
\IfFileExists{microtype.sty}{%
\usepackage{microtype}
\UseMicrotypeSet[protrusion]{basicmath} % disable protrusion for tt fonts
}{}
\usepackage[margin=1in]{geometry}
\usepackage{hyperref}
\hypersetup{unicode=true,
            pdftitle={Reading reflections},
            pdfauthor={Shen Qu},
            pdfborder={0 0 0},
            breaklinks=true}
\urlstyle{same}  % don't use monospace font for urls
\usepackage{graphicx,grffile}
\makeatletter
\def\maxwidth{\ifdim\Gin@nat@width>\linewidth\linewidth\else\Gin@nat@width\fi}
\def\maxheight{\ifdim\Gin@nat@height>\textheight\textheight\else\Gin@nat@height\fi}
\makeatother
% Scale images if necessary, so that they will not overflow the page
% margins by default, and it is still possible to overwrite the defaults
% using explicit options in \includegraphics[width, height, ...]{}
\setkeys{Gin}{width=\maxwidth,height=\maxheight,keepaspectratio}
\IfFileExists{parskip.sty}{%
\usepackage{parskip}
}{% else
\setlength{\parindent}{0pt}
\setlength{\parskip}{6pt plus 2pt minus 1pt}
}
\setlength{\emergencystretch}{3em}  % prevent overfull lines
\providecommand{\tightlist}{%
  \setlength{\itemsep}{0pt}\setlength{\parskip}{0pt}}
\setcounter{secnumdepth}{0}
% Redefines (sub)paragraphs to behave more like sections
\ifx\paragraph\undefined\else
\let\oldparagraph\paragraph
\renewcommand{\paragraph}[1]{\oldparagraph{#1}\mbox{}}
\fi
\ifx\subparagraph\undefined\else
\let\oldsubparagraph\subparagraph
\renewcommand{\subparagraph}[1]{\oldsubparagraph{#1}\mbox{}}
\fi

%%% Use protect on footnotes to avoid problems with footnotes in titles
\let\rmarkdownfootnote\footnote%
\def\footnote{\protect\rmarkdownfootnote}

%%% Change title format to be more compact
\usepackage{titling}

% Create subtitle command for use in maketitle
\providecommand{\subtitle}[1]{
  \posttitle{
    \begin{center}\large#1\end{center}
    }
}

\setlength{\droptitle}{-2em}

  \title{Reading reflections}
    \pretitle{\vspace{\droptitle}\centering\huge}
  \posttitle{\par}
  \subtitle{USP 570}
  \author{Shen Qu}
    \preauthor{\centering\large\emph}
  \postauthor{\par}
      \predate{\centering\large\emph}
  \postdate{\par}
    \date{Week 3}


\begin{document}
\maketitle

\hypertarget{main-point}{%
\subsubsection{Main point}\label{main-point}}

\begin{itemize}
\tightlist
\item
  Transportation system and spatial form
\end{itemize}

Muller (2017) reviews the evolution of the U.S. urban form and describe
the four eras of intrametropolitan growth inludes walking-horsecar era,
electric streetcar era, recreational auto era, and freeway era. We can
see the transportation technology is a determining constraint to other
factors for urban form. The four-stage urban transportation development
have their dominated spatial structure, which cannot be represented by
some socio-economic factors. In the last section of his paper wrote in
1995, Muller stated the two problem of congestion and spatial mismatch
caused by suburbanization and auto dependency. He also summarized some
socioeconomic dynamics such as postindustrial economy, globalization,
and the expansion of the services sector. He didn't talk about how new
transportation techonology may launch the next era of metroplitian
expansion. Now we can see some emerging techonological breakthrough is
happening. Intelligent Transportation Systems (ITS) are replaceing
precious travel decision mechanism. Many scholars start to predict the
new urban forms affeceted by autononous vehicles. However, admitting
techonolgy as a initial force cannont tel us how will it forge a
decidedly different future. As Levinson and Krizek (2018) emphasise
transportation is a neccessary but not a sufficient factor for any
development.

\begin{itemize}
\tightlist
\item
  The choice of house and job
\end{itemize}

Levinson and Krizek (2018 ch.3) introduce several theories and models
explaining how transport influences residential location preferences.
From an economic perspective, the transport cost is the core element in
Thünen's model of agricultural land and Alonso's Bid-rent theory.
Meanwhile, Schelling's Segregation Model and Tiebout's model of `vote
with feet' disclose the significant function of self-selection as a
social element.

Levinson and Krizek (2018 ch.4) point out ``the theory of behavior based
on gravity models assume that geograohy plays a prominent role in
predicting who interacts with whom and how frequently.'' This theory can
explain the macro structure. Some evidences such as `excess driving' and
`weak ties' show that transportation network is only a part of travel
decisions. Social networks with a `hub-and-spoke structure' play a
prominent role in finding a job.

The mechanism of social networks is difficult to define and messure by
such as `weak or strong link', `close relation or not'. We have to
depend on modeling to connnect the phenomenon and the roots. It is
important to highlight the complex and indirect role of grography from
Levinson and Krizek (2018)'s argument.

\hypertarget{discussion}{%
\subsubsection{Discussion}\label{discussion}}

In describing the process of housing choice, some subjectively assessed
attributes are hardly measured. Hedonic regression analysis is a method
using modeling techniques for measuring non-market benefits uses the
change in prices of complementary goods to infer a willing to pay (WTP)
for a healthier environment or less risky environment. The main
hypothesis is that the accessibility of transit or bike facility has
some positive effects on housing sale prices. Here we compare the
methodology and findings from three articles using hedonic models (Chen,
Rufolo, and Dueker 1998; Welch, Gehrke, and Wang 2016; Liu and Shi
2017).

\begin{itemize}
\tightlist
\item
  The models
\end{itemize}

All the three studies used ordinary least-squares (OLS) modeling, which
based on maximum likelihood principle. Chen, Rufolo, and Dueker (1998)`s
model\footnote{Chen, Rufolo, and Dueker (1998)'s model:
  \[P=a+bX +rZ +e\] where \(P\) the sales price or log transformation,
  \(a\) constant term, \(X\) a vector of control variables, \(Z\) a
  vector of spatial-related variables, \(e\) the random error term, and
  \(a, b, r\) parameters to be estimated.} is a typical multiple
regression. Both Liu and Shi (2017)\footnote{Liu and Shi (2017)'s model
  \[P_i=\beta_0+\beta_1T_i+\beta_2H_i+\beta_3R_i+\beta_4B_i+\varepsilon_i\]
  \(P_i\) Property sale price; \(T_i\) TransacSon characterisScs, such
  as year and season of the sale; \(H_i\) Internal property
  characterisScs , such as age, size and property tax liability; \(R_i\)
  External neighborhood characterisScs, such as school quality, crime
  rate, and walk score; \(B_i\) Bike facility characterisScs, such as
  distance to nearest advanced bicycle facility, and advanced bike
  facility density within a half-mile radius.
  \[\begin{cases}Y=\rho W Y+X\beta+\varepsilon& \text{Spatial lag model}\\Y=X\beta+\lambda W\varepsilon+\nu& \text{Spatial error model} \end{cases}\]
  where \(\rho W Y\) spatially lagged dependent variable that represents
  the omitted variable in the regression model, \(\rho\) spatial lag
  parameter, \(W\) spatial weighting matrix that represents the
  interaction between different locations, and \(X\) vector of all
  variables included in the OLS model. \(\lambda\) spatial error
  parameter, \(W\varepsilon\) spatial error, interpreted as the mean
  error from neighboring locations, and \(\nu\) independent model error.}
and Welch, Gehrke, and Wang (2016)\footnote{Welch, Gehrke, and Wang
  (2016)'s spatial panel data model
  \[\begin{cases}y=\lambda(I_T\cdot W_N)y+X\beta+u\\u=(\iota_T\cdot I_N)\mu+\varepsilon&\text{sum of the temporal autocorrelation} \\\varepsilon=\rho(I_T\cdot W_N)\varepsilon+\upsilon&\text{spatial autocorrelation} \end{cases}\]
  where, \(y\) is an \(NT\times1\) vector of observations on the
  dependent variable, \(X\) is a \(NT\times k\) matrix of observations
  on the non-stochastic exogenous variables, \(I_T\) is an identity
  matrix of dimension \(T\), \(W_N\) is an \(N\times N\) spatial weights
  matrix with diagonal elements set to zero, \(\lambda\) represents the
  corresponding spatial parameter.} try to control spatial dependence
and prevent overestimation of coefficient estimates. Welch, Gehrke, and
Wang (2016) employ a a spatiotemporal autocorrelation model to overcome
the potentially confounding modeling errors. To avoid sample bias,
Welch, Gehrke, and Wang (2016)'s spatial panel data model adopt a
bootstrapping regression estimates with a nonparametric approach, which
constructed a 'pseudo spatial panel dataset' from a single observation
for every grid cell, each year of the study period.

Chen, Rufolo, and Dueker (1998) examine the simple effects of LRT. The
other two research conside the corelated effects of improved bike and
rail transit facility access. Chen, Rufolo, and Dueker (1998) evaluate
the combined effect of accessibility and nuisance respond the distance
from station or line. However, for LRT, the positive effect of
accessibility is only related with the station, not the line. Welch,
Gehrke, and Wang (2016) use street network distance instead of the
straight-line measurements or a series of areal buffer approximations.

`Long term' means observe the cumulated effects by cross-sectional data
analysis. Welch, Gehrke, and Wang (2016) think panel dataset can examine
the long-term influence better. Since the opening of the Yellow Line in
2004, Green Line in 2009, and Central Loop Line in 2012, is DID method
comparing the pre- and post-treatment effects from transit investments
better?

Chen, Rufolo, and Dueker (1998) only studied a corridor of the light
rail system (MAX) in Portland, Oregon. The other two studies include all
the City of Portland. With this advantage, Liu and Shi (2017) can
examine the extensiveness of the bike network and Welch, Gehrke, and
Wang (2016) create a 300-meter grid cell system cast over the city.

Levinson and Krizek (2018) divide the home attributes to three
catagories: structural (including internal and external) attributes,
location attributes, and neighborhood characteristics. All three studies
basically follow this division. Moreover, Liu and Shi (2017) distinguish
the two property types of SFHs and MFHs, add the advanced bicycle
facility characteristics. Her study measures bothease of access
(distance) and extensiveness of bike network (density). Welch, Gehrke,
and Wang (2016) examine the bike facilities types,including local and
regional, on-street and off-street. There is also a risk of overfitting,
which making an overly complex model to explain idiosyncrasies in the
data under study.

Some relevant hedonic price studies show some consistent findings.
Al-Mosaind et al. (1993), Lewis-Workman and Brod (1997), Chen et al.
(1998), Dueker and Bianco (1999), Welch, Gehrke, and Wang (2016) found
the positive effects by Light rail transit. Lindsey et al. (2004) Krizek
(2006) Asabere and Huffman (2009) Parent and vom Hofe (2013) found
Multi-use paths have positive effects. Krizek (2006) and Welch, Gehrke,
and Wang (2016) found the effects of bike lanes are not significant or
negative. Liu and Shi (2017) further found extensiveness of the bike
network is a positive and statistically significant contributor to
property prices after controlling for proximity to bike facilites and
other internal and external variables. As more studies on different
urban regions, a meta analysis may be neccessary.

\hypertarget{notes}{%
\section{Notes}\label{notes}}

\hypertarget{references}{%
\section*{References}\label{references}}
\addcontentsline{toc}{section}{References}

\hypertarget{refs}{}
\leavevmode\hypertarget{ref-chen1998measuring}{}%
Chen, Hong, Anthony Rufolo, and Kenneth J Dueker. 1998. ``Measuring the
Impact of Light Rail Systems on Single-Family Home Values: A Hedonic
Approach with Geographic Information System Application.''
\emph{Transportation Research Record} 1617 (1). SAGE Publications Sage
CA: Los Angeles, CA: 38--43. \url{https://doi.org/10.3141/1617-05}.

\leavevmode\hypertarget{ref-levinson2018metropolitan}{}%
Levinson, David M, and Kevin J Krizek. 2018. \emph{Metropolitan Land Use
and Transport: Planning for Place and Plexus}. Routledge.
\url{https://doi.org/10.4324/9781315684482}.

\leavevmode\hypertarget{ref-litman2017evaluating}{}%
Litman, Todd. 2017. \emph{Evaluating Accessibility for Transport
Planning}. Victoria Transport Policy Institute.
\url{http://www.vtpi.org/access.pdf}.

\leavevmode\hypertarget{ref-liu2017impact}{}%
Liu, Jenny H, and Wei Shi. 2017. ``Impact of Bike Facilities on
Residential Property Prices.'' \emph{Transportation Research Record}
2662 (1). SAGE Publications Sage CA: Los Angeles, CA: 50--58.
\url{https://doi.org/10.3141/2662-06}.

\leavevmode\hypertarget{ref-Muller2017transportation}{}%
Muller, Peter O. 2017. ``Transportation and Urban Form.'' In \emph{The
Geography of Urban Transportation, Fourth Edition}, edited by G.
Giuliano and S. Hanson, 57--85. Guilford Publications.
\url{https://books.google.com/books?id=J3GnDQAAQBAJ}.

\leavevmode\hypertarget{ref-welch2016long}{}%
Welch, Timothy F, Steven R Gehrke, and Fangru Wang. 2016. ``Long-Term
Impact of Network Access to Bike Facilities and Public Transit Stations
on Housing Sales Prices in Portland, Oregon.'' \emph{Journal of
Transport Geography} 54. Elsevier: 264--72.
\url{https://doi.org/10.1016/j.jtrangeo.2016.06.016}.


\end{document}
