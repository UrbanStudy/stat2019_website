\documentclass[12pt,]{article}
\usepackage[]{mathpazo}
\usepackage{amssymb,amsmath}
\usepackage{ifxetex,ifluatex}
\usepackage{fixltx2e} % provides \textsubscript
\ifnum 0\ifxetex 1\fi\ifluatex 1\fi=0 % if pdftex
  \usepackage[T1]{fontenc}
  \usepackage[utf8]{inputenc}
\else % if luatex or xelatex
  \ifxetex
    \usepackage{mathspec}
  \else
    \usepackage{fontspec}
  \fi
  \defaultfontfeatures{Ligatures=TeX,Scale=MatchLowercase}
\fi
% use upquote if available, for straight quotes in verbatim environments
\IfFileExists{upquote.sty}{\usepackage{upquote}}{}
% use microtype if available
\IfFileExists{microtype.sty}{%
\usepackage{microtype}
\UseMicrotypeSet[protrusion]{basicmath} % disable protrusion for tt fonts
}{}
\usepackage[margin=0.5in]{geometry}
\usepackage{hyperref}
\hypersetup{unicode=true,
            pdfborder={0 0 0},
            breaklinks=true}
\urlstyle{same}  % don't use monospace font for urls
\usepackage{graphicx,grffile}
\makeatletter
\def\maxwidth{\ifdim\Gin@nat@width>\linewidth\linewidth\else\Gin@nat@width\fi}
\def\maxheight{\ifdim\Gin@nat@height>\textheight\textheight\else\Gin@nat@height\fi}
\makeatother
% Scale images if necessary, so that they will not overflow the page
% margins by default, and it is still possible to overwrite the defaults
% using explicit options in \includegraphics[width, height, ...]{}
\setkeys{Gin}{width=\maxwidth,height=\maxheight,keepaspectratio}
\IfFileExists{parskip.sty}{%
\usepackage{parskip}
}{% else
\setlength{\parindent}{0pt}
\setlength{\parskip}{6pt plus 2pt minus 1pt}
}
\setlength{\emergencystretch}{3em}  % prevent overfull lines
\providecommand{\tightlist}{%
  \setlength{\itemsep}{0pt}\setlength{\parskip}{0pt}}
\setcounter{secnumdepth}{0}
% Redefines (sub)paragraphs to behave more like sections
\ifx\paragraph\undefined\else
\let\oldparagraph\paragraph
\renewcommand{\paragraph}[1]{\oldparagraph{#1}\mbox{}}
\fi
\ifx\subparagraph\undefined\else
\let\oldsubparagraph\subparagraph
\renewcommand{\subparagraph}[1]{\oldsubparagraph{#1}\mbox{}}
\fi

%%% Use protect on footnotes to avoid problems with footnotes in titles
\let\rmarkdownfootnote\footnote%
\def\footnote{\protect\rmarkdownfootnote}

%%% Change title format to be more compact
\usepackage{titling}

% Create subtitle command for use in maketitle
\providecommand{\subtitle}[1]{
  \posttitle{
    \begin{center}\large#1\end{center}
    }
}

\setlength{\droptitle}{-2em}

  \title{}
    \pretitle{\vspace{\droptitle}}
  \posttitle{}
    \author{}
    \preauthor{}\postauthor{}
    \date{}
    \predate{}\postdate{}
  

\begin{document}

\begin{enumerate}
\def\labelenumi{\arabic{enumi}.}
\tightlist
\item
  Assume that \(X_1,X_2,..X_{10}\) is a random sample from a
  distribution having a p.d.f. of the form
  \(f(x)=\begin{cases}\lambda x^{\lambda-1}& 0<x<1\\0&\text{otherwise}\end{cases}\)
\end{enumerate}

\[f(x)=\lambda x^{\lambda-1}=\lambda e^{(\lambda-1)\ln x}\implies\ln x\sim Exp()\]

Let \(Y_i=-\ln x_i\), \(0<x<1\), then \(X=e^{-y}\),
\(\frac{dx}{dy}=-e^{-y}\)

\(g(y)=\lambda(e^{-y})^{\lambda-1}|-e^{-y}|=\lambda e^{-\lambda y}\sim Exp(\lambda), y>0\)

So
\(\sum_{i=1}^{10}Y_i=-\sum_{i=1}^{10}\ln x_i\sim Gamma(\alpha=10,\beta=\frac1{\lambda})=\chi^2_{20}\),
then,

\[\Lambda=\frac{\sup L(\hat\lambda_0|x)}{\sup L(\hat\lambda_{1}|x)}=\frac{(1/2)^{10}(\prod_{i=1}^{10} x_i)^{1/2-1}}{(1)^{10}(\prod_{i=1}^{10} x_i)^{1-1}}=(1/2)^{10}(\prod_{i=1}^{10} x_i)^{-1/2}\le C\]

Take derivative of both sides, \(\Lambda\le C\) is equivalent
\(-\sum_{i=1}^{10}\ln x_i\le C'_1\) or
\(-\sum_{i=1}^{10}\ln x_i\ge C'_2\).

\(\chi^2_{(0.05/2),20}=34.16961\), \(\chi^2_{(1-0.05/2),20}=9.590777\)

So the critical region are \(-\sum_{i=1}^{10}\ln x_i\in(0, 9.590777]\)
and \(-\sum_{i=1}^{10}\ln x_i\in[34.16961,\infty)\)

\begin{center}\rule{0.5\linewidth}{\linethickness}\end{center}

\[L(\lambda)=\lambda^{10}(\prod_{i=1}^{10} x_i)^{\lambda-1}=\lambda^{10}e^{(\lambda-1)\sum^{10}_{i=1} \ln x_i}\]
\[l(\lambda)=10\ln\lambda+(\lambda-1)\sum^{10}_{i=1} \ln x_i\]
\[l'(\lambda)=\frac{10}\lambda-\sum^{10}_{i=1} \ln x_i\overset{\text{set}}{=}0\]

\[\hat\lambda_{MLE}=\frac{10}{\sum_{i=1}^{10}\ln x_i}\]
\[E[\hat\lambda_{MLE}]=10E[Y^{-1}]=10\frac{\beta^{-1}\Gamma(\alpha-1)}{\Gamma(\alpha)}=\frac{10\lambda\Gamma(10-1)}{\Gamma(10)}=\frac{10\lambda}{9}\]

\begin{center}\rule{0.5\linewidth}{\linethickness}\end{center}

\begin{enumerate}
\def\labelenumi{\arabic{enumi}.}
\setcounter{enumi}{4}
\tightlist
\item
  Find the MUVE of \(\lambda\).
\end{enumerate}

\begin{itemize}
\tightlist
\item
  \textbf{Step1: Proof sufficient}
\end{itemize}

From \emph{Fisher--Neyman factorization theorem} (\texttt{2019-2-14p5})

\[f(x|\lambda)=L(\lambda)=\lambda^n(\prod_{i=1}^n x_i)^{\lambda-1}=\lambda^ne^{(\lambda-1)\sum^n_{i=1} \ln x_i}\cdot1=k(t|\lambda)h(\vec x)\]

\(h(\vec x)=1\) is free of \(\lambda\). So \(T=\sum^n_{i=1} \ln x_i\) is
a sufficient statistic for \(\lambda\).

\begin{itemize}
\tightlist
\item
  \textbf{Step2: Proof complete}
\end{itemize}

\(f(x|\lambda)\) is a member of the exponential family
(\texttt{2019-2-19p12}). By the Theorem of Complete Statistics in the
exponential family

\[f(x|\vec\lambda)=\lambda^ne^{\sum^n_{i=1} (\lambda-1)\ln x_i}=h(x)c(\vec \lambda)e^{\sum^k_{j=1}W_j(\vec \lambda)t_j(x)}\]

For pdf \(f(x)>0\) and \(x^{\lambda-1}>0\), \(\lambda>0\).
\(\{W_1(\vec \lambda),..,W_k(\vec \lambda)\}\) contains an open interval
in \(\Bbb R\), so \(T(\vec x)=\sum^n_{i=1} \ln x_i)\) is a complete
sufficient statistic for \(\lambda\).

\begin{enumerate}
\def\labelenumi{\arabic{enumi}.}
\setcounter{enumi}{5}
\tightlist
\item
  Show that the MUVE of \(\lambda\) is asymptotically efficient.
\end{enumerate}

\begin{center}\rule{0.5\linewidth}{\linethickness}\end{center}

\begin{enumerate}
\def\labelenumi{\alph{enumi}.}
\setcounter{enumi}{1}
\tightlist
\item
  Find the expected value of \(\hat\lambda_{MLE}\).
\end{enumerate}

\begin{itemize}
\item
  Method 1
\item
  Method 2
\end{itemize}

By 2.1.10 Probability integral transformation, let
\(U_i=F_X(\mathbf{x}|\lambda)=\int_{-\infty}^x\lambda x^{-\lambda-1}dx=1-\int_1^{x}\lambda x^{-\lambda-1}dx=1-(\left.-x^{-\lambda}\right|_{1}^x)=x^{-\lambda} \sim Uni(0,1)\),

By 5.6.3 the exponential-uniform transformation,
\(\sum_{i=1}^n\ln X=-\frac1\lambda\sum_{i=1}^n\ln U_i\sim Gamma(n,\frac1\lambda)\);
\((\sum_{i=1}^n\ln x_i)^{-1}\sim Inv-Gamma(n,\frac1\lambda)\).

For a Inv-Gamma\((\alpha,\beta)\),
\(f_{X}(x)=\frac{\beta^{\alpha}}{\Gamma(\alpha)}x^{-\alpha-1}e^{-\frac\beta{x}},x>0\),
\(E[x^n]=\frac{\beta^n}{(\alpha-1)\cdots(\alpha-n)}\).

Thus,
\[E[\hat\lambda]=E\left[\frac{n}{\sum_{i=1}^n\ln x_i}\right]=nE\left[(\sum_{i=1}^n\ln x_i)^{-1}\right]=\frac{n\lambda}{n-1}\]
----

\begin{enumerate}
\def\labelenumi{\alph{enumi}.}
\setcounter{enumi}{2}
\tightlist
\item
  Find the variance of \(\hat\lambda_{MLE}\).
\end{enumerate}

From method 1,

\[E[\hat\lambda^2]=n^2E[Y^{-2}]=n^2\frac{\beta^{-2}\Gamma(-2+\alpha)}{\Gamma(\alpha)}=\frac{n^2\lambda^2\Gamma(n-2)}{\Gamma(n)}=\frac{n^2\lambda^2(n-3)!}{(n-1)!}=\frac{n^2\lambda^2}{(n-1)(n-2)}\]

From method 2,

For Inv-Gamma\((\alpha,\beta)\),
\(E[x^n]=\frac{\beta^n}{(\alpha-1)\cdots(\alpha-n)}\), then

\[E[\hat\lambda^2]=E\left[\frac{n^2}{(\sum_{i=1}^n\ln x_i)^2}\right]=n^2E\left[(\sum_{i=1}^n\ln x_i)^{-2}\right]=\frac{n^2\lambda^2}{(n-1)(n-2)}\]

Therefore,
\[Var[\hat\lambda^2]=\frac{n^2\lambda^2}{(n-1)(n-2)}-\frac{n^2\lambda^2}{(n-1)^2}=\frac{n^2\lambda^2}{(n-1)}[\frac1{n-2}-\frac1{n-1}]=\frac{n^2\lambda^2}{(n-1)^2(n-2)}\]

\begin{center}\rule{0.5\linewidth}{\linethickness}\end{center}

\begin{enumerate}
\def\labelenumi{\alph{enumi}.}
\setcounter{enumi}{3}
\tightlist
\item
  Using \(\hat\lambda_{MLE}\), create an unbiased estimator
  \(\hat\lambda_{U}\).
\end{enumerate}

\(E[\hat\lambda]=\frac{n\lambda}{n-1}\) is a biased estimator.

We can set
\(\frac{n-1}{n}E[\hat\lambda]=E[\frac{n-1}{n}\cdot\frac{n\lambda}{n-1}]=\lambda\)

Therefore, \(E[\hat\lambda_{U}]=E[\frac{n-1}{n}E[\hat\lambda]]=\lambda\)

\[\hat\lambda_{U}=\frac{n-1}{n}\hat\lambda_{MLE}\quad \text{is an unbiased estimator.}\]

\begin{center}\rule{0.5\linewidth}{\linethickness}\end{center}

\begin{enumerate}
\def\labelenumi{\alph{enumi}.}
\setcounter{enumi}{4}
\tightlist
\item
  Find the variance of \(\hat\lambda_{U}\).\texttt{2019-3-5p12}
\end{enumerate}

\[Var[\hat\lambda_{U}]=Var[\frac{n-1}{n}\hat\lambda_{MLE}]=(\frac{n-1}{n})^2\frac{n^2\lambda^2}{(n-1)^2(n-2)}=\frac{\lambda^2}{n-2}\]


\end{document}
