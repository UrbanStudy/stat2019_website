\documentclass[12pt,]{article}
\usepackage[]{mathpazo}
\usepackage{amssymb,amsmath}
\usepackage{ifxetex,ifluatex}
\usepackage{fixltx2e} % provides \textsubscript
\ifnum 0\ifxetex 1\fi\ifluatex 1\fi=0 % if pdftex
  \usepackage[T1]{fontenc}
  \usepackage[utf8]{inputenc}
\else % if luatex or xelatex
  \ifxetex
    \usepackage{mathspec}
  \else
    \usepackage{fontspec}
  \fi
  \defaultfontfeatures{Ligatures=TeX,Scale=MatchLowercase}
\fi
% use upquote if available, for straight quotes in verbatim environments
\IfFileExists{upquote.sty}{\usepackage{upquote}}{}
% use microtype if available
\IfFileExists{microtype.sty}{%
\usepackage{microtype}
\UseMicrotypeSet[protrusion]{basicmath} % disable protrusion for tt fonts
}{}
\usepackage[margin=1in]{geometry}
\usepackage{hyperref}
\hypersetup{unicode=true,
            pdftitle={Reading reflections},
            pdfborder={0 0 0},
            breaklinks=true}
\urlstyle{same}  % don't use monospace font for urls
\usepackage{graphicx,grffile}
\makeatletter
\def\maxwidth{\ifdim\Gin@nat@width>\linewidth\linewidth\else\Gin@nat@width\fi}
\def\maxheight{\ifdim\Gin@nat@height>\textheight\textheight\else\Gin@nat@height\fi}
\makeatother
% Scale images if necessary, so that they will not overflow the page
% margins by default, and it is still possible to overwrite the defaults
% using explicit options in \includegraphics[width, height, ...]{}
\setkeys{Gin}{width=\maxwidth,height=\maxheight,keepaspectratio}
\IfFileExists{parskip.sty}{%
\usepackage{parskip}
}{% else
\setlength{\parindent}{0pt}
\setlength{\parskip}{6pt plus 2pt minus 1pt}
}
\setlength{\emergencystretch}{3em}  % prevent overfull lines
\providecommand{\tightlist}{%
  \setlength{\itemsep}{0pt}\setlength{\parskip}{0pt}}
\setcounter{secnumdepth}{0}
% Redefines (sub)paragraphs to behave more like sections
\ifx\paragraph\undefined\else
\let\oldparagraph\paragraph
\renewcommand{\paragraph}[1]{\oldparagraph{#1}\mbox{}}
\fi
\ifx\subparagraph\undefined\else
\let\oldsubparagraph\subparagraph
\renewcommand{\subparagraph}[1]{\oldsubparagraph{#1}\mbox{}}
\fi

%%% Use protect on footnotes to avoid problems with footnotes in titles
\let\rmarkdownfootnote\footnote%
\def\footnote{\protect\rmarkdownfootnote}

%%% Change title format to be more compact
\usepackage{titling}

% Create subtitle command for use in maketitle
\newcommand{\subtitle}[1]{
  \posttitle{
    \begin{center}\large#1\end{center}
    }
}

\setlength{\droptitle}{-2em}

  \title{Reading reflections}
    \pretitle{\vspace{\droptitle}\centering\huge}
  \posttitle{\par}
    \author{}
    \preauthor{}\postauthor{}
      \predate{\centering\large\emph}
  \postdate{\par}
    \date{Due April 11}


\begin{document}
\maketitle

\hypertarget{section}{%
\subsubsection{\texorpdfstring{\textcolor[rgb]{0.7,0.7,0.7}{Week 2}}{}}\label{section}}

\begin{enumerate}
\def\labelenumi{\arabic{enumi}.}
\tightlist
\item
  Levinson \& Krizek - Chapters 1 \& 2
\end{enumerate}

\begin{itemize}
\tightlist
\item
  \textbf{Accessibility}
\end{itemize}

\begin{enumerate}
\def\labelenumi{\arabic{enumi}.}
\setcounter{enumi}{1}
\item
  OECD, International Transport Forum, Linking People and Places, 2017
  \url{https://www.itf-oecd.org/linking-people-and-places}
\item
  Susan Handy, ``Enough with the ``Ds'' Already --- Let's Get Back to
  ``A'' Transfers Magazine, Spring 2018
  \url{https://transfersmagazine.org/enough-with-the-ds-already-lets-get-back-to-a/}
\end{enumerate}

当Robert Cervero和Kara
Kockelman在1997年发表他们被高度引用的文章``旅行需求和3D:密度,多样性和设计''时,他们似乎永远地改变了交通规划的话语。使用恰好以字母D开头的三种措施来表征建筑环境的想法很吸引人,并且抓住了它。

到2010年,3D已经增长到7,增加了``目的地可访问性''(我称之为区域可访问性),``距离传输'',``需求管理''和``人口统计''(一类控制变量,而不是建筑环境的特征)。从那时起,我听说有一两个D或两个传闻。

Ds现在在学术文献中占据了建筑环境对旅行行为的影响。我认为这不是一件好事。

该术语至少令人困惑。``多样性''是指特定区域内土地利用的混合,但人们可能很容易将其误认为是该地区居民的社会人口组合,而不是``人口统计数据''。``设计''通常被视为连通性街道网络,而不是这个词通常所暗示的街道环境的审美品质。Ds是一个引人注目的速记,但不是特别清楚。

另一个问题是研究人员将D特征视为独立的,而实际上它们是相互依赖的。研究人员通常会仔细测试``多重共线性'',即变量高度相互关联的情况,因此难以估计其中任何一个变量对另一个变量的影响。大多数研究表明D特征并不像人们想象的那样相关。例如,有可能在街道连通性较低的地区拥有良好的土地利用组合。

但是,D特征的某些值经常在一起,反映了特定邻域形成的时代。例如,二战前的社区更有可能同时拥有网格街道网络(``设计'')和步行距离内的社区规模购物(``多样性'')。

Ds也倾向于一起,因为它们相互影响。密度尤其对其他D特征具有强烈影响:更高的密度支持更好的运输服务(``距离运输'')以及更接近零售和服务(``多样性'')。更大的土地利用``多样性'',加上更好的街道连通性(``设计''),可以缩短到目的地的距离,从而提高``目的地可达性''。

将Ds视为彼此独立处理会产生过高估计其影响力的风险。它还会造成低估其影响力的风险。如果两个或多个特征一起具有协同效应,则它们产生的总效应大于每个独立效应的总和。

研究人员很少讨论这些关系,更不用说在他们的分析中对它们进行说明了。特别令人不安的是使用密度作为旅行行为的预测因子,而没有解释
密度如何
影响行为。密度影响其他D特征,更直接地决定旅行者可用的选择,从而更直接地塑造他们的旅行选择。研究人员通常引用其他研究人员对相同措施的依赖来证明他们对密度和其他Ds的依赖。即使是基于行为理论而非公认实践来证明Ds合理性的最小尝试也是一种改进。

如果我们完全改变了框架怎么办?

让我们从旅行者的角度出发,以及她如何看待自己的旅行选择。旅行行为研究人员通常认为,个人可以选择日常旅行,最大限度地发挥其效用,提供最大的成本效益,或者更简单地说,使个人最有意义。

建筑环境在确定个人可用选择方面发挥着作用。最根本的是,建筑环境特征,如密度,土地使用组合和街道连通性决定了个人离目的地的距离,克服这个距离的成本会影响她可以去哪里,通过什么模式,以及如何经常。如果目标是了解旅行者做出的选择,那么有什么更好的方式来表征建筑环境而不是它提供的选择?

可访问性的概念提供了完美的方法。如通常所定义的,来自给定地点的可访问性水平反映了其周围的目的地的分布,各种模式可以容易地达到这些目的地,以及在那里发现的活动的数量和特征。它告诉我们建筑环境为旅行者提供的选择。

研究表明,可达性和旅行行为之间存在密切联系。在一项研究中,我们发现到最近的商店的距离是在那里散步的强烈预测。另一方面,在半英里范围内拥有更多商店与更频繁地步行到商店有关。在第三项研究中,我们发现进入一个步行距离内有商店的街区和良好的交通导致车辆行驶里程减少。这些结果在行为上都是完美的。

从研究的角度来看,可访问性不仅是一个更好的衡量标准。从实践的角度来看,这也是一个更好的衡量标准。对人们来说最重要的是他们到达他们需要的地方是多么容易,以及访问他们需要或想要的服务是多么容易。

城市本身不会提高密度 -
它们会提高密度以增加可达性。使用可访问性作为衡量当前条件和拟议政策的绩效衡量标准,可能会使公众辩论摆脱密度的可怕想法,而密度往往引发敌对的反应。

为了实现良好的可达性目标,德国人有一个简单的(虽然难以发音)短语: ein
stadt de kuerzen
wegen,一个短距离的城市。这是一个几乎每个人都可以达成共识的目标,它为一系列可以减少自身依赖和提高生活质量的策略打开了大门。

可访问性可能不如Ds那么吸引人,但它对于研究人员,从业者和公众来说更有意义。现在是时候完成Ds并回到A.

\begin{enumerate}
\def\labelenumi{\arabic{enumi}.}
\setcounter{enumi}{3}
\tightlist
\item
  One article from the list on the first tab of the shared Google sheet
  for this course:
\end{enumerate}

\url{https://docs.google.com/spreadsheets/d/1G83Ar37TttNj6MD6Hj13-2HMaXgjqvQwauKjrYWlqDU/}

For the article, pay particular attention to the theories being use to
guide the research. You will be expected to discuss your article in
class, explaining the theory or theories used and how land use/built
environment was measured and incorporated into the analysis. Refer to
pages 4-5 in Levinson \& Krizek for a brief overview of some of the
relevant theories.


\end{document}
