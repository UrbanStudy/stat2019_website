\documentclass[12pt,]{article}
\usepackage{lmodern}
\usepackage{amssymb,amsmath}
\usepackage{ifxetex,ifluatex}
\usepackage{fixltx2e} % provides \textsubscript
\ifnum 0\ifxetex 1\fi\ifluatex 1\fi=0 % if pdftex
  \usepackage[T1]{fontenc}
  \usepackage[utf8]{inputenc}
\else % if luatex or xelatex
  \ifxetex
    \usepackage{mathspec}
  \else
    \usepackage{fontspec}
  \fi
  \defaultfontfeatures{Ligatures=TeX,Scale=MatchLowercase}
    \setmainfont[]{Times New Roman}
\fi
% use upquote if available, for straight quotes in verbatim environments
\IfFileExists{upquote.sty}{\usepackage{upquote}}{}
% use microtype if available
\IfFileExists{microtype.sty}{%
\usepackage{microtype}
\UseMicrotypeSet[protrusion]{basicmath} % disable protrusion for tt fonts
}{}
\usepackage[margin=1in]{geometry}
\usepackage{hyperref}
\hypersetup{unicode=true,
            pdftitle={Reading reflections},
            pdfauthor={Shen Qu},
            pdfborder={0 0 0},
            breaklinks=true}
\urlstyle{same}  % don't use monospace font for urls
\usepackage{longtable,booktabs}
\usepackage{graphicx,grffile}
\makeatletter
\def\maxwidth{\ifdim\Gin@nat@width>\linewidth\linewidth\else\Gin@nat@width\fi}
\def\maxheight{\ifdim\Gin@nat@height>\textheight\textheight\else\Gin@nat@height\fi}
\makeatother
% Scale images if necessary, so that they will not overflow the page
% margins by default, and it is still possible to overwrite the defaults
% using explicit options in \includegraphics[width, height, ...]{}
\setkeys{Gin}{width=\maxwidth,height=\maxheight,keepaspectratio}
\IfFileExists{parskip.sty}{%
\usepackage{parskip}
}{% else
\setlength{\parindent}{0pt}
\setlength{\parskip}{6pt plus 2pt minus 1pt}
}
\setlength{\emergencystretch}{3em}  % prevent overfull lines
\providecommand{\tightlist}{%
  \setlength{\itemsep}{0pt}\setlength{\parskip}{0pt}}
\setcounter{secnumdepth}{0}
% Redefines (sub)paragraphs to behave more like sections
\ifx\paragraph\undefined\else
\let\oldparagraph\paragraph
\renewcommand{\paragraph}[1]{\oldparagraph{#1}\mbox{}}
\fi
\ifx\subparagraph\undefined\else
\let\oldsubparagraph\subparagraph
\renewcommand{\subparagraph}[1]{\oldsubparagraph{#1}\mbox{}}
\fi

%%% Use protect on footnotes to avoid problems with footnotes in titles
\let\rmarkdownfootnote\footnote%
\def\footnote{\protect\rmarkdownfootnote}

%%% Change title format to be more compact
\usepackage{titling}

% Create subtitle command for use in maketitle
\providecommand{\subtitle}[1]{
  \posttitle{
    \begin{center}\large#1\end{center}
    }
}

\setlength{\droptitle}{-2em}

  \title{Reading reflections}
    \pretitle{\vspace{\droptitle}\centering\huge}
  \posttitle{\par}
  \subtitle{USP 570}
  \author{Shen Qu}
    \preauthor{\centering\large\emph}
  \postauthor{\par}
      \predate{\centering\large\emph}
  \postdate{\par}
    \date{Week 9}


\begin{document}
\maketitle

\begin{itemize}
\tightlist
\item
  The `Diamond of Operation'
\end{itemize}

Levinson and Krizek (2018 Chapter.12) provides a \(2\times2\)
combination to explain the transportation administering. The two row are
Queuing and charging. As the author said, ``Based on ability and
willingness to wait, queues ration makes people wait in line,'' which is
more equitable. ``Prices ration based on ability and willingness to pay
money, are more efficient than queses from a economic point of view.''
The two columns are short term and long term. When ``operating and
allocating scarce resources'', the mechanisms between short term and
long term are different. The short-term strategies can allocate use of
scarce road space through queuing or pricing. The long-term strategies
focus on growth controls for funding infrastructure or achieving other
goals like revitalization, smart growth, etc.

\begin{itemize}
\tightlist
\item
  Growth management and funding transportation
\end{itemize}

Levinson and Krizek (2018) uses another \(2\times2\) catergories to
summarize the techniques of value capture: New or existing land
development ; New or exsiting infrastructure. The new-new combination
includes Impact fees and Joint development; the double-existing
combination has Land-value taxes and Transportation utility fees; The
new infrastructure with existing development use Special assessments and
Tax increment financing (TIF); The new development with existing
infrastructure relates to Air right.

Vadali et al. (2018) disscussed these techniques from a perspective of
investment. The guidebook introduced ten value capture methods with two
categories: Land Value return and recycling methods and Land value
return-like methods. The former is ``the public recovery of a portion of
the increased land value that is created as a result of public-sector
investment in infrastructure and the reutilization of that value to
invest in infrastructure,'' which is based on the beneficiary principle.
The later is based on the cost principle.

\begin{longtable}[]{@{}ll@{}}
\toprule
\begin{minipage}[b]{0.52\columnwidth}\raggedright
Land value return and recycling methods\strut
\end{minipage} & \begin{minipage}[b]{0.37\columnwidth}\raggedright
Land value return--like methods\strut
\end{minipage}\tabularnewline
\midrule
\endhead
\begin{minipage}[t]{0.52\columnwidth}\raggedright
Land value tax or split rate tax\strut
\end{minipage} & \begin{minipage}[t]{0.37\columnwidth}\raggedright
Transportation utility fee\strut
\end{minipage}\tabularnewline
\begin{minipage}[t]{0.52\columnwidth}\raggedright
Betterment levy\strut
\end{minipage} & \begin{minipage}[t]{0.37\columnwidth}\raggedright
Tax increment financing\strut
\end{minipage}\tabularnewline
\begin{minipage}[t]{0.52\columnwidth}\raggedright
Special assessment district fee\strut
\end{minipage} & \begin{minipage}[t]{0.37\columnwidth}\raggedright
Development impact fee\strut
\end{minipage}\tabularnewline
\begin{minipage}[t]{0.52\columnwidth}\raggedright
Sale of public land or air rights\strut
\end{minipage} & \begin{minipage}[t]{0.37\columnwidth}\raggedright
Exaction or proffer\strut
\end{minipage}\tabularnewline
\begin{minipage}[t]{0.52\columnwidth}\raggedright
Lease of public land or air rights\strut
\end{minipage} & \begin{minipage}[t]{0.37\columnwidth}\raggedright
\strut
\end{minipage}\tabularnewline
\begin{minipage}[t]{0.52\columnwidth}\raggedright
Joint development fee or interface fee\strut
\end{minipage} & \begin{minipage}[t]{0.37\columnwidth}\raggedright
\strut
\end{minipage}\tabularnewline
\bottomrule
\end{longtable}

\begin{itemize}
\tightlist
\item
  Discussion:
\end{itemize}

The short-term strategies usually arrange the traffic flow to optimize
the use of transportation infrastructure. (Levinson and Krizek 2018)
Through congestion or charging, the transportation engineers try to
achieve the maximum flow or eliminate the bottleneck. Some policies or
tools may have significant effects on transportation in short term but
may come back to its original point in long run. Such as, adding a new
lane for a highway may release the congestion for several month or
several years. But it will attact more traffic flow from around here,
influence people's location choices, induce more demand, and resume the
current congestion level at the end.

Urban planner will be more concerned about The long-term strategies,
which think of solving the transportation prbloms through regulating the
use of land. The concept behind the growth management is that
``providing public goods and services creates value and those who
receive that value should return a portion of that value to the public
sector to compensate for the costs incurred to provide the public goods
and services.'' Because transportation is a semi-public good with the
properties of nonrivalrous and nonexcludable. Up til now, there is not
efficint way to capture and measure the use of transporation
infrastructure for each user. Even it becomes possible, the charging
cost and transaction cost are unacceptable. By the development of ITS
and automated driving system, a realtime full-coverage transportation
services may be provided for everyone in the future. As claimed by some
TNC, the transportation-as-a-service (Taas) can precisely calculate the
degree of resouse consuming for each trip, including vehicles, roads,
environmental effects, and management system. The customers just choose
time, destinations, quality, and payment. This scenario will redefine
the boundary of public goods in transportation field. Once the new
mechanism can internalize all the exernalities, the land use may have a
whole new pattern and distinctive spatial structure.

\hypertarget{references}{%
\section*{References}\label{references}}
\addcontentsline{toc}{section}{References}

\hypertarget{refs}{}
\leavevmode\hypertarget{ref-levinson2018metropolitan}{}%
Levinson, David M, and Kevin J Krizek. 2018. \emph{Metropolitan Land Use
and Transport: Planning for Place and Plexus}. Routledge.
\url{https://doi.org/10.4324/9781315684482}.

\leavevmode\hypertarget{ref-vadali2018guidebook}{}%
Vadali, Sharada, Johanna Zmud, Todd Carlson, Karin DeMoors, Rick Rybeck,
Steven Fitzroy, Naomi Stein, and Mark Sieber. 2018. \emph{Guidebook to
Funding Transportation Through Land Value Return and Recycling}. Project
19-13. \url{https://doi.org/10.17226/25110}.


\end{document}
