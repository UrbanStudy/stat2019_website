\documentclass[]{article}
\usepackage{lmodern}
\usepackage{amssymb,amsmath}
\usepackage{ifxetex,ifluatex}
\usepackage{fixltx2e} % provides \textsubscript
\ifnum 0\ifxetex 1\fi\ifluatex 1\fi=0 % if pdftex
  \usepackage[T1]{fontenc}
  \usepackage[utf8]{inputenc}
\else % if luatex or xelatex
  \ifxetex
    \usepackage{mathspec}
  \else
    \usepackage{fontspec}
  \fi
  \defaultfontfeatures{Ligatures=TeX,Scale=MatchLowercase}
\fi
% use upquote if available, for straight quotes in verbatim environments
\IfFileExists{upquote.sty}{\usepackage{upquote}}{}
% use microtype if available
\IfFileExists{microtype.sty}{%
\usepackage{microtype}
\UseMicrotypeSet[protrusion]{basicmath} % disable protrusion for tt fonts
}{}
\usepackage[margin=1in]{geometry}
\usepackage{hyperref}
\hypersetup{unicode=true,
            pdftitle={USP537 Notes},
            pdfborder={0 0 0},
            breaklinks=true}
\urlstyle{same}  % don't use monospace font for urls
\usepackage{longtable,booktabs}
\usepackage{graphicx,grffile}
\makeatletter
\def\maxwidth{\ifdim\Gin@nat@width>\linewidth\linewidth\else\Gin@nat@width\fi}
\def\maxheight{\ifdim\Gin@nat@height>\textheight\textheight\else\Gin@nat@height\fi}
\makeatother
% Scale images if necessary, so that they will not overflow the page
% margins by default, and it is still possible to overwrite the defaults
% using explicit options in \includegraphics[width, height, ...]{}
\setkeys{Gin}{width=\maxwidth,height=\maxheight,keepaspectratio}
\IfFileExists{parskip.sty}{%
\usepackage{parskip}
}{% else
\setlength{\parindent}{0pt}
\setlength{\parskip}{6pt plus 2pt minus 1pt}
}
\setlength{\emergencystretch}{3em}  % prevent overfull lines
\providecommand{\tightlist}{%
  \setlength{\itemsep}{0pt}\setlength{\parskip}{0pt}}
\setcounter{secnumdepth}{0}
% Redefines (sub)paragraphs to behave more like sections
\ifx\paragraph\undefined\else
\let\oldparagraph\paragraph
\renewcommand{\paragraph}[1]{\oldparagraph{#1}\mbox{}}
\fi
\ifx\subparagraph\undefined\else
\let\oldsubparagraph\subparagraph
\renewcommand{\subparagraph}[1]{\oldsubparagraph{#1}\mbox{}}
\fi

%%% Use protect on footnotes to avoid problems with footnotes in titles
\let\rmarkdownfootnote\footnote%
\def\footnote{\protect\rmarkdownfootnote}

%%% Change title format to be more compact
\usepackage{titling}

% Create subtitle command for use in maketitle
\providecommand{\subtitle}[1]{
  \posttitle{
    \begin{center}\large#1\end{center}
    }
}

\setlength{\droptitle}{-2em}

  \title{USP537 Notes}
    \pretitle{\vspace{\droptitle}\centering\huge}
  \posttitle{\par}
    \author{}
    \preauthor{}\postauthor{}
      \predate{\centering\large\emph}
  \postdate{\par}
    \date{Winter 2019}

\usepackage{multicol}

\begin{document}
\maketitle

\begin{multicols}{3}

#  

## I. Introduction to Transportation Economics

### Economics problems



 - How to allocating scarce resources  
 - Between transportation and other uses  
 - Between different transportation modes  
 - Scarcity Choice Opportunity Costs  
 
\end{multicols}

\hypertarget{transport-problems-and-economic-theory}{%
\subsubsection{Transport Problems and Economic
Theory}\label{transport-problems-and-economic-theory}}

\begin{itemize}
\tightlist
\item
  Consumer surplus\\
\item
  Cost allocation models\\
\item
  Marginal cost pricing principle\\
\item
  Discrete choice models
\end{itemize}

\hypertarget{history-of-transit}{%
\subsubsection{History of Transit}\label{history-of-transit}}

\begin{itemize}
\tightlist
\item
  Prior to mid-1800s: Walk
\item
  Mid-1800s to 1900: Horse-drawn rail
\item
  1900: Electric railways
\item
  1910s: Motor buses
\item
  1970s: Decline of LRT
\item
  1970s: Congress approves capital grants for new rail starts
\end{itemize}

\hypertarget{history-of-the-road-system}{%
\subsubsection{History of the Road
System}\label{history-of-the-road-system}}

1890s: interest by railroads, farmers, autos pushed for roads 1916:
Federal Aid System started with \$75 million for highway improvements
over 5 years 1919: First state gas tax implemented 1930s: Highway
improvements accelerated increased technology in trucks \& buses 1944:
State Interstate Highway System 1945: State highway system (90\%
Federally financed)

Trends in VMT per capita

Mode Share Trends

Mode Share

\hypertarget{transportation-public-policy}{%
\subsubsection{Transportation Public
Policy}\label{transportation-public-policy}}

\begin{itemize}
\tightlist
\item
  Subsidies

  \begin{itemize}
  \tightlist
  \item
    Urban Transit (capital \& operating)
  \item
    Highway Distortions lead to
  \item
    Problems: shifted land use

    \begin{itemize}
    \tightlist
    \item
      Opportunity cost patterns
    \item
      Cost-effectiveness
    \item
      Underestimation of capital cost of building and maintenance
      Distortions lead to shifted land use patterns
    \end{itemize}
  \end{itemize}
\item
  Public ownership

  \begin{itemize}
  \tightlist
  \item
    Potential problems:
  \item
    Reduced productivity and cost-effectiveness
  \item
    Regulated high supply
  \end{itemize}
\end{itemize}

Criteria for economic evaluation

\begin{itemize}
\item
  Efficiency
\item
  Cost-effectiveness
\item
  Equity
\end{itemize}

\hypertarget{microeconomic-foundations}{%
\subsection{Microeconomic Foundations}\label{microeconomic-foundations}}

\hypertarget{microeconomics-review}{%
\subsubsection{Microeconomics review}\label{microeconomics-review}}

\begin{itemize}
\item
  Supply \& Demand
\item
  Efficiency MC = MR TC = FC + VC TR = P x Q Profit = TR -- TC
  Monopolist profit-maximization; deadweight loss
\item
  Elasticities: Price elasticity of demand Income elasticity of demand
  Cross-price elasticity of demand
\item
  Demand

  \begin{itemize}
  \item
    Law of demand: the quantity demanded of a good falls when the price
    of the good rises, other things equal; downward sloping
  \item
    Derived from consumer utility maximization theory
  \item
    Demand curve represents MB (marginal benefit)
  \item
    Shifts in demand
  \item
    Transportation demand

    \begin{itemize}
    \item
      Derived demand
    \item
      Time-specific
    \end{itemize}
  \end{itemize}
\item
  Supply
\item
  Law of supply: an increase in price results in an increase in quantity
  supplied, all else equal
\item
  Derived from producer profit maximization theory
\item
  Supply curve represents MC (marginal cost)
\item
  Shifts in supply
\item
  Elasticities:
\item
  Price elasticity of demand: percentage change of quantity demanded in
  response to a 1\% change in price; price sensitivity of consumer
  demand
\end{itemize}

\[E=\frac{\text{Percentage change in }Qd}{\text{Percentage change in }P}=\frac{\frac{\Delta Q}Q}{\frac{\Delta P}P}\]

\begin{itemize}
\item
  Income elasticity of demand
\item
  Cross-price elasticity of demand
\item
  Demand: MB = 10 -- Q
\item
  Supply: MC = Q
\item
  Equilibrium occurs at MB = MC
\item
  Q* = 5 (PS)
\item
  P* (market clearing price) = \$5
\item
  What is consumer surplus (CS), producer surplus (PS) and total surplus
  (net benefit)?
\item
  Efficiency \& Market Equilibrium for Monopolist
\item
  Demand: \(P=5–\frac{Q}{1000}\)
\item
  Fixed cost: \(FC=\$5000\) (each day)
\item
  Operating cost is \(\$50\) per hour, and can provide 100 rides
\item
  What is the MC? \(MC=\$\frac{50}{100} = \$0.50\)
\item
  Total cost: \(TC=FC+VC=5000+0.5Q\)
\item
  Total Revenue:
  \(TR=P\times Q=(5–\frac{Q}{1000})Q=5Q–\frac{Q^2}{1000}\)
\end{itemize}

\hypertarget{critical-issues-in-transportation}{%
\subsubsection{Critical Issues in
Transportation}\label{critical-issues-in-transportation}}

\begin{itemize}
\tightlist
\item
  Congestion
\item
  Energy, environment and climate change
\item
  Infrastructure
\item
  Finance
\item
  Equity
\item
  Emergency preparedness, response and mitigation
\item
  Safety
\item
  Institutions
\item
  Human and intellectual capital
\item
  Reliability \& resilience of transportation systems
\item
  Safety
\item
  Energy, environment and climate
\item
  Infrastructure funding
\end{itemize}

\hypertarget{microeconomics-review-1}{%
\subsubsection{Microeconomics Review}\label{microeconomics-review-1}}

\begin{itemize}
\item
  Demand: shifts caused by income, price of other goods and services
  (complements and substitutes), tastes/trends, expectation of future
  price changes
\item
  Demand for transport is a derived demand
\item
  It is time-specific
\item
  Follows peaks and troughs throughout day, season, week, etc.
\item
  Supply:
\end{itemize}

shifts caused by cost of production, government policy, price of related
goods, natural shocks, technology

\begin{itemize}
\tightlist
\item
  Elasticities:
\end{itemize}

Effects of changes in tolls, transit fares, congestion pricing, fuel
price On VMT, \# of cars, \# of trips, trips on public transit, fuel
consumption, vehicle fuel economy

\begin{itemize}
\tightlist
\item
  Example:
\end{itemize}

Demand: \(P=5–\frac{Q}{1000}\)

Fixed cost: \(FC = \$5000\) (each day)

Operating cost is \(\$50\) per hour, and can provide 100 rides what is
the MC? \(MC = \$\frac{50}{100}=\$0.50\)

Total cost: \(TC = FC + VC = 5000 + 0.5Q\)

Total Revenue: \(TR=P\times Q=(5–\frac{Q}{1000})Q=5Q–\frac{Q^2}{1000}\)

a). What is the equilibrium number of rides if price is set at MC? MR =
MC (marginal revenue is P in perfect competition!)

\(5–\frac{Q}{1000}=0.50\)

\(Q^*=500\)

What is the TC? TR? Profit?

\(TC=5000+0.5(4500)=5000+2250=7250\)

\(TR=0.50\times 4500=2250\)

\(Profit=2250–7250=-\$5000\) (negative!! Which equates to a subsidy for
a publicly owned agency)

b). Elasticity between (P,Q) = (1,4000) to (2, 3000)? How to interpret
it?

c). If the operator is a monopolist, what is the P, Q, profit and
deadweight loss?

MR = derivative of \(TR=5–\frac{Q}{500}\) (law of demand)

MR = MC to decide profit-maximizing quantity ; \(5–\frac{Q}{500}=0.50\)
; \(Q_m=2250\)

P is determined with demand function;
\(P = 5 – \frac{2250}{1000} = \$2.75\)

\(DWL=0.5\times2.25\times1750=\$1968.75\)

\(Profit=(2.75\times2250)-(5000+0.5\times2250)=\$62.50\)

\hypertarget{ii.-demand-for-transportation}{%
\subsection{II. Demand for
Transportation}\label{ii.-demand-for-transportation}}

\hypertarget{types-of-demand}{%
\paragraph{Types of Demand}\label{types-of-demand}}

\begin{itemize}
\tightlist
\item
  When to travel
\item
  To what destination
\item
  What mode of transportation
\item
  What route to take
\end{itemize}

\hypertarget{choices-impacting-transportation-demand}{%
\paragraph{Choices impacting transportation
demand}\label{choices-impacting-transportation-demand}}

\begin{itemize}
\tightlist
\item
  Decisions made by organizations
\item
  Decisions made by households or individuals
\item
  Complexities
\item
  Interrelated decisions
\item
  Spatial and temporal
\item
  Quality of service
\item
  Derived demand
\item
  Supply and demand interactions via congestion
\item
  Heterogeneity
\end{itemize}

\hypertarget{transportation-demand-and-economic-development}{%
\paragraph{Transportation demand and Economic
Development}\label{transportation-demand-and-economic-development}}

\begin{itemize}
\tightlist
\item
  ``Chicken or the egg''?
\item
  Supply led view transportation; economic development
\item
  Demand led view Economic development; transportation
\end{itemize}

\hypertarget{theory-of-consumer-choice}{%
\paragraph{Theory of Consumer Choice}\label{theory-of-consumer-choice}}

\url{https://en.wikipedia.org/wiki/Consumer_choice}

\begin{itemize}
\tightlist
\item
  Continuous utility function (X1=VMT; X2=other goods) --increasing \&
  quasi-concave
\item
  Consumer utility maximization problem
\item
  Derivation of income elasticity of demand
\item
  Derivation of demand curve
\item
  Price changes; substitution \& income effect
\end{itemize}

\hypertarget{aggregate-demand-models}{%
\subsubsection{Aggregate Demand Models}\label{aggregate-demand-models}}

\hypertarget{income-elasticities-of-motor-vehicle-ownership}{%
\paragraph{Income elasticities of motor vehicle
ownership}\label{income-elasticities-of-motor-vehicle-ownership}}

\texttt{p.331-336\ Income\ Elasticities\ of\ Motor\ Vehicle\ Ownership\ and\ Use}

\begin{itemize}
\tightlist
\item
  Table 10-2. Income Elasticities of Motor Vehicle Ownership and Usage
\end{itemize}

Four stylized facts emerge in the estimates shown in table 10-2. +
First, income elasticities from time-series data are typically smaller
than those from cross-section data. That is because cross-section
analyses produce long run elasticities, and long-run behavior is
generally more responsive to income changes than short-run behavior.
Roughly speaking, long-run income elasticities of motor vehicle
(especially car) ownership are greater than 1.0, while short-run income
elasticities are less than 1.0. + Second, income elasticities from
urban-level data are similar to or smaller than those from country-level
data, largely because there are more competing modes of transportation
and greater congestion in urban areas, both of which reduce the
attraction of motor vehicles. In fact, the long-run elasticities from
urban-level data are closer to those from country-level data than are
the shortrun elasticities. + Third, income elasticities are generally
larger for automobiles than for commercial vehicles, supporting the
economic hypothesis that the share of passenger cars in the motor
vehicle fleet increases with income. + Finally, income elasticities of
motor vehicle use are less than unity, smaller than long-run income
elasticities of motor vehicle ownership, and smaller for households with
one vehicle than for those with two, indicating that motor vehicle use
increases less rapidly than motor vehicle ownership.

These findings also support the earlier economic hypotheses.

\begin{itemize}
\tightlist
\item
  Figure 10-1. Per Capita Income and Motor Vehicle Ownership in Fifty
  Countries and Thirty-Five Cities
\end{itemize}

figure 10-1 indicates that motor vehicle ownership increases somewhat
more rapidly with income at the national level than at the urban level.

\begin{itemize}
\tightlist
\item
  Vehicle ownership
\item
  Population density elasticity of vehicle ownership= -0.4 (urban) \&
  -0.1 (national)
\item
  Spatial spread of economic activity
\item
  Urban: density motorization
\item
  National: urbanization income
\end{itemize}

\hypertarget{income-elasticities-of-road-length}{%
\paragraph{Income elasticities of road
length}\label{income-elasticities-of-road-length}}

\begin{itemize}
\tightlist
\item
  Road length
\item
  F(size of economy, geographical area, population, income per capital,
  population density)
\item
  Population density elasticity of road length
\end{itemize}

\texttt{p.339-344\ Road\ Provision}

\begin{itemize}
\tightlist
\item
  Figure 10-2. Per Capita Income and Per Capita Road Length, Fifty
  Countries and Thirty-Five Cities
\end{itemize}

The size of the national road network is associated with the size of the
economy, geographical area, population, income per capita, and
population density.

Per capita income is a major determinant of road length at the national
level.

Both paved and total road length increase at a constant rate with per
capita income, as can be seen in figure 10-2.

Estimates at the national level (using the techniques employed for
vehicle ownership) find no saturation level for road density with
respect to per capita income.

\begin{itemize}
\tightlist
\item
  Table 10-4. Estimated Effects of Population, Per Capita Income, and
  Population Density on Road Length
\end{itemize}

Paved road length has an elasticity of 1.0 with respect to income (when
income increases by 1 percent, paved road length increases by 1
percent), while overall road length increases only about half as fast as
income (table 10-4).

Population is a significant determinant of national total and paved road
length, whereas population density affects only total road length.

The major implication of these findings is that at the national level,
paved road length increases with per capita income at roughly the same
rate as vehicle ownership. As a result, congestion does not appear to be
a current or growing problem for the national road network in most
countries.

\hypertarget{case-study-1-demand-for-gas-and-automobile-travel}{%
\paragraph{Case Study 1 Demand for gas and automobile
travel}\label{case-study-1-demand-for-gas-and-automobile-travel}}

\begin{itemize}
\item
  Gt = f(Real Gas Price, Real Personal Income, Population, Season,
  Gasoline Crisis) Hypotheses:
\item
  Law of demand
\item
  Opportunity cost = real gas price + queuing cost
\item
  Gasoline is normal good
\item
  Market demand is horizontal sum of individual demands
\end{itemize}

\hypertarget{case-study-2-demand-for-urban-rail-rapid-transit}{%
\paragraph{Case Study \#2 -- Demand for urban rail rapid
transit}\label{case-study-2-demand-for-urban-rail-rapid-transit}}

\begin{itemize}
\tightlist
\item
  Estimation results
\end{itemize}

\hypertarget{disaggregate-demand-models}{%
\subsubsection{Disaggregate Demand
Models}\label{disaggregate-demand-models}}

\hypertarget{random-utility-model-rum}{%
\paragraph{Random Utility Model (RUM)}\label{random-utility-model-rum}}

\begin{itemize}
\tightlist
\item
  Systematic component
\item
  Random component
\end{itemize}

\texttt{p.15\ The\ Random\ Utility\ Framework}

\[U_j = V(X_j, S;\beta) + \varepsilon_j \]

\begin{itemize}
\tightlist
\item
  Example: Two choices -- auto or bus
\end{itemize}

If \(U_{bus} > U_{auto}\), choose bus If \(U_{bus} < U_{auto}\), choose
auto

\hypertarget{binomial-probit}{%
\paragraph{Binomial Probit}\label{binomial-probit}}

\texttt{p.17\ Binary\ Probit\ and\ the\ Value\ of\ Travel\ Time}

\begin{itemize}
\tightlist
\item
  Lave (1969) estimation on Chicago commuters:
\end{itemize}

\[V=-2.08D^T-0.00759w\cdot t-0.0186c-0.0254(Inc\cdot Dist\cdot D^T)+ 0.0255(Age\cdot D^T)-0.057(Female\cdot D^T)\]

where DT is an alternative-specific dummy variable equal to 1 for
transit and 0 for auto. It enters the model independently (in the first
term) and also interacts with the traveler's income (Inc), trip distance
(Dist), age, and a dummy variable indicating whether the traveler is
female. The traveler's wage rate is denoted by w, travel time by t, and
travel cost by c. Note that \(D^T\), t, and c all vary from one mode to
the other, whereas w, Inc, Dist, Age, and Female do not. Lave's more
detailed results show that all estimated parameters are statistically
significant except the last. The model indicates that travelers are less
likely to take transit if their income or trip distance increases, but
more likely to take transit as they become older.

This utility function is linear in travel time and cost. The value of
travel time (VOT), defined as the marginal rate of substitution between
time and cost, is just the ratio of the time and cost coefficients of
that linear relation:

\[VOT =\frac{\delta V}{\delta t}\frac{\delta c}{\delta V}=\frac{-0.00759w}{-0.0186} = 0.41w\]

Lave's finding, which is representative of other estimates in the
literature, is that time is valued at 41 percent of the average wage
rate. Note that the variables in this model were specified so that VOT
is proportional to the wage rate. This approach is consistent with
models of time allocation, which suggest that a person's trade-off
between travel time and money is strongly related to his or her
possibilities for earning money in the labor market.

\hypertarget{logit}{%
\paragraph{Logit}\label{logit}}

\texttt{p.19\ Multinomial\ Logit\ and\ Urban\ Mode\ Choice}

\begin{itemize}
\tightlist
\item
  McFadden et al. (1977)
\end{itemize}

multinomial logit for urban work trips

auto, bus/walk, bus/auto, carpool

\[V = -0.0412c/w – 0.0201t – 0.0531t^0 – 0.89 D^1 – 1.78 D^3 – 2.15 D^4\]

\begin{itemize}
\tightlist
\item
  What is the probability of each mode (mode share)?
\end{itemize}

\[P_i=\frac{e^{V_i}}{\sum_{j=1}^Je^{V_j}}\]

\begin{itemize}
\tightlist
\item
  Assumption: Independence from irrelevant alternatives (IIA)
\end{itemize}

\hypertarget{logit-example}{%
\paragraph{Logit Example}\label{logit-example}}

\[V = -0.0412c/w – 0.0201t – 0.0531t^0 – 0.89 D^1 – 1.78 D^3 – 2.15 D^4\]

What are the mode splits estimated by this model?

\hypertarget{nested-logit}{%
\paragraph{Nested Logit}\label{nested-logit}}

\begin{itemize}
\item
  Lam and Small (2001)
\item
  Choices not hierarchical
\item
  Simultaneous
\item
  Allows correlation between choices in same nest
\end{itemize}

\[V = -0.874 D^{tag}+ 0.0239 IncD^{tag}–0.767 ForLangD^{tag}–0.785 D^{lane}–0.356c –0.150t –0.217R + \text{other terms}\]

\hypertarget{accessibility-and-economic-opportunity}{%
\subsection{Accessibility and Economic
Opportunity}\label{accessibility-and-economic-opportunity}}

``post-war changes in urban structure and urban transportation systems
have conferred significant improvements and greater satisfactions on the
majority, {[}but{]} they almost certainly have caused a relative
deterioration in the access to opportunities, if not in the actual
mobility of a significant fraction of the poor.'' --Meyer (1968)

\hypertarget{spatial-mismatch-hypothesis}{%
\subsubsection{Spatial Mismatch
Hypothesis}\label{spatial-mismatch-hypothesis}}

\begin{itemize}
\tightlist
\item
  Kain(1968):
\end{itemize}

Employment opportunities; Workers

\begin{itemize}
\tightlist
\item
  Concentration of jobs increased in suburbs
\end{itemize}

white households dispersed to suburban areas while black households
remained in central cities

black households are now located far away from suburban jobs

\begin{itemize}
\item
  Suburbanization of jobs (Fujita and Thisse2002; Cerveroet al.~2002;
  Iceland and Harris 1998)
\item
  Disconnection of black households from jobs (Martin 2001; Raphael and
  Stoll 2002; Stoll 2006)
\end{itemize}

\hypertarget{factors-of-spatial-mismatch-hypothesis}{%
\subsubsection{Factors of Spatial Mismatch
Hypothesis}\label{factors-of-spatial-mismatch-hypothesis}}

\begin{itemize}
\tightlist
\item
  Gobillonet al (2007)
\item
  Commuting costs are high
\item
  Job search efficiency decreases with distance to jobs (information)
\item
  Distant workers search less intensively
\item
  High search costs cause workers to restrict spatial search
\item
  Employer discrimination: redlining or statistical discrimination
\item
  Employer refuse to hire distant workers
\item
  Customer discrimination
\end{itemize}

\hypertarget{policy-directions}{%
\subsubsection{Policy Directions}\label{policy-directions}}

\begin{itemize}
\tightlist
\item
  Bring jobs to people
\item
  Bring people to jobs
\item
  Connecting people to jobs
\end{itemize}

\hypertarget{iii.-transportation-and-land-use}{%
\subsection{III. Transportation and Land
Use}\label{iii.-transportation-and-land-use}}

\hypertarget{general-concepts}{%
\paragraph{General concepts}\label{general-concepts}}

\begin{itemize}
\tightlist
\item
  R = annual rental payment for land (reflecting highest private use
  value of land)
\item
  Market Value = R / interest rate
\item
  Land rent is input cost of production
\end{itemize}

\hypertarget{transportation-land-use}{%
\paragraph{Transportation \& Land Use}\label{transportation-land-use}}

\begin{itemize}
\item
  Bid-rent function: the willingness-to-pay (WTP) for a piece of land
  for a given level of profit (per acre)
\item
  Firm location choice
\item
  Household location choice
\end{itemize}

\hypertarget{location-equilibrium-in-monocentriccity}{%
\paragraph{Location equilibrium in
monocentriccity}\label{location-equilibrium-in-monocentriccity}}

\hypertarget{application-streetcar}{%
\subsubsection{Application: Streetcar}\label{application-streetcar}}

\hypertarget{relative-locations-before-streetcar}{%
\paragraph{Relative locations before
streetcar}\label{relative-locations-before-streetcar}}

\hypertarget{effect-1-reduced-commute-cost}{%
\paragraph{Effect \#1: Reduced commute
cost}\label{effect-1-reduced-commute-cost}}

\hypertarget{effect-2-1-reduced-wage-effects-on-residential-bid-rents}{%
\paragraph{Effect \#2-1: Reduced wage effects on residential bid
rents}\label{effect-2-1-reduced-wage-effects-on-residential-bid-rents}}

\hypertarget{effect-2-2-reduced-wage-effects-on-business-bid-rents}{%
\paragraph{Effect \#2-2: Reduced wage effects on business bid
rents}\label{effect-2-2-reduced-wage-effects-on-business-bid-rents}}

\hypertarget{transportation-land-use-1}{%
\paragraph{Transportation ; Land use}\label{transportation-land-use-1}}

\texttt{p.\ 412-413\ Residential\ Development}

Figure 12-1. Density of New Residential Development by Decade for
Selected Cities

\hypertarget{technologies-and-their-effects-on-travel-speeds-and-times}{%
\paragraph{Technologies and their effects on travel speeds and
times}\label{technologies-and-their-effects-on-travel-speeds-and-times}}

Table 12-1. Effects of Technological Innovations on Travel Speeds and
Times

\hypertarget{international-comparisons}{%
\paragraph{International comparisons}\label{international-comparisons}}

\hypertarget{iv.-project-evaluation}{%
\subsection{IV. Project Evaluation}\label{iv.-project-evaluation}}

\begin{itemize}
\item
  Cost-benefit analysis (CBA)
\item
  Pareto improvements?
\item
  Potential Pareto improvements (Kaldor-Hicks Criteria)
\item
  Why perform CBA for transportation choices?
\item
  Limited funding for transportation investments
\item
  Linkages between transportation investments and land use, economic,
  environmental and energy policy concerns
\item
  Net Social Benefit
\item
  NSB = Benefit -- Cost
\end{itemize}

\begin{enumerate}
\def\labelenumi{\arabic{enumi}.}
\tightlist
\item
  Specify alternatives/define situation
\item
  Decide whose benefits and costs count (standing)
\item
  Identify the impact categories (benefits or costs) \& measurement
  indicators
\item
  Predict the quantitative impacts over the life
\item
  Monetize impacts
\item
  Discount benefits \& costs to obtain present values
\item
  Compute NPV of each alternative
\item
  Perform sensitivity analysis
\item
  Make recommendation
\end{enumerate}

\hypertarget{define-the-situation-scope}{%
\paragraph{\#1-\#2: Define the situation \&
scope}\label{define-the-situation-scope}}

\begin{itemize}
\tightlist
\item
  Changes in resource allocation
\item
  Comparing between alternatives
\item
  What resources are at stake?
\item
  What jurisdiction should CBA be done for?
\end{itemize}

\hypertarget{costs}{%
\paragraph{\#3: Costs}\label{costs}}

\begin{itemize}
\tightlist
\item
  Survey Approach
\item
  Engineering Approach
\item
  Combined Approach
\end{itemize}

\hypertarget{evaluating-costs}{%
\paragraph{\#3: Evaluating Costs}\label{evaluating-costs}}

\begin{itemize}
\tightlist
\item
  Price mechanism true social value/cost
\item
  Shadow price: estimated value of an input or output to the economy or
  society as a whole
\item
  Marginal opportunity cost -- value of the best alternative use of the
  input
\item
  Related to market price, but not exactly the same
\end{itemize}

\hypertarget{costs-vs-transfers}{%
\paragraph{\#3: Costs vs Transfers}\label{costs-vs-transfers}}

\begin{itemize}
\tightlist
\item
  When resources are not used up nor created, but just shifted from one
  set of individuals to another, we say the resources are transferred.
\item
  Transfers
\item
  Do not change total resources in the community
\item
  Changes distribution of the resources
\end{itemize}

\hypertarget{benifits}{%
\paragraph{\#3: Benifits}\label{benifits}}

\begin{itemize}
\tightlist
\item
  Types of benefits
\item
  Resources measurable in monetary terms: goods and services sold in the
  market places Price
\item
  Resources measurable in physical units, but not in monetary terms.
\item
  Resources valued by the community, but not measurable by any means.
\end{itemize}

\hypertarget{benifits-vs-transfers}{%
\paragraph{\#3: Benifits vs Transfers}\label{benifits-vs-transfers}}

\begin{itemize}
\tightlist
\item
  {[}EX{]} Are jobs a benefit?
\item
  If the economy is at full-employment (which is often the assumption of
  CBA), the opportunity cost of using workers is that they are not used
  somewhere else.
\item
  Thus, the new jobs are simply a transfer of workers from one sector to
  another.
\item
  Locally: net gains in local employment?
\end{itemize}

\hypertarget{valuing-benefits}{%
\paragraph{\#3: Valuing Benefits}\label{valuing-benefits}}

\begin{itemize}
\tightlist
\item
  Types of Values
\item
  Use Value
\item
  Passive use value
\item
  Option Value
\item
  Nonuse Value
\item
  Existence value
\item
  Bequest value
\end{itemize}

\hypertarget{valuing-benefits---methods}{%
\paragraph{\#3: Valuing Benefits -
Methods}\label{valuing-benefits---methods}}

\begin{itemize}
\tightlist
\item
  Revealed Preference Method -- valuing based on actual observable
  choices
\item
  Market Prices: Market demand reflects the consumer's WTP, so prices
  can be used to evaluate loss of value
\item
  Benefits Transfer
\item
  Meta-analysis
\item
  Indirect:
\item
  Travel cost method
\item
  hedonic property values / hedonic wages
\end{itemize}

\hypertarget{revealed-preference-benefits-transfer}{%
\paragraph{Revealed Preference -- Benefits
Transfer}\label{revealed-preference-benefits-transfer}}

\begin{itemize}
\tightlist
\item
  Benefits Transfer Approach
\item
  The application of existing information \& knowledge to new contexts
\item
  Useful when collecting primary data and analysis is impractical (cost
  or time)
\item
  3 important features
\item
  Policy context must be well-defined
\item
  Data must meet certain criteria
\item
  Study site and new site should correspond
\item
  Limitations
\end{itemize}

\hypertarget{indirect-revealed-preference-hedonic-analysis}{%
\paragraph{Indirect Revealed Preference -- Hedonic
Analysis}\label{indirect-revealed-preference-hedonic-analysis}}

\begin{itemize}
\tightlist
\item
  Hedonic Analysis -- Uses the changes in prices of related goods to
  infer a WTP for a healthier environment or less risky environment
  (with multiple regression analysis)
\item
  Hedonic Property Value
\item
  Hedonic Wage Method
\end{itemize}

\hypertarget{indirect-revealed-preference-hedonic-property-value}{%
\paragraph{Indirect Revealed Preference -- Hedonic Property
Value}\label{indirect-revealed-preference-hedonic-property-value}}

\begin{itemize}
\tightlist
\item
  Hedonic Property Value Example (Mendelsohn et al.~1992)
\item
  PCB contamination in New Bedford, MA
\item
  Compare change in prices for houses sold before and after
  contamination became public. Control for all other factors affecting
  home costs.
\item
  Houses closest to the contaminated area: price declines of \$9,000; in
  area of secondary pollution, declines of \$7,000. Total damages to
  home-owners: \$36 million
\item
  Firms paid out at least \$20 million in natural resource damage claims
\end{itemize}

\hypertarget{indirect-revealed-preference-hedonic-wage-method}{%
\paragraph{Indirect Revealed Preference -- Hedonic Wage
Method}\label{indirect-revealed-preference-hedonic-wage-method}}

\begin{itemize}
\tightlist
\item
  Hedonic Wage Method -- similar to hedonic property value, but attempts
  to isolate the component of wages that serve to compensate workers in
  risky occupations for taking on the risk
\end{itemize}

\begin{longtable}[]{@{}lll@{}}
\toprule
Occupation & Wage (hourly) & Risk of Death
(statistically)\tabularnewline
\midrule
\endhead
Backhoe operator & \$15 & .0001\tabularnewline
Bulldozer & \$16 & .00015\tabularnewline
Grader operator & \$17 & .0002\tabularnewline
Lawnmower & \$18 & .00025\tabularnewline
\bottomrule
\end{longtable}

Plot expected earnings over lifetime of workers for various occupations
and then extrapolate from the estimated line to get VSL

\begin{itemize}
\tightlist
\item
  Problems
\item
  Accurate information?
\item
  Sample selection bias
\item
  Involuntary nature of risk
\item
  Are labor market choices to accept higher risk really ``choices''?
\item
  Unpaid workers
\end{itemize}

\hypertarget{value-of-a-statistical-life-vsl}{%
\paragraph{Value of a Statistical Life
(VSL)}\label{value-of-a-statistical-life-vsl}}

\begin{itemize}
\item
  EPA: \$7.8 million (2008) \$6.9 million
\item
  Water Division (EPA): \$8.7 million
\item
  Viscusi meta-analysis study: \$3 -- 7 million
\item
  EX: Suppose there is regulation costing \$18 billion to enforce but
  will prevent 2,500 deaths. How much is spent to save every life?
\end{itemize}

\$18 billion / 2500 = \$7.2 million per life saved

Would it have passed EPA's CBA in 2006 or 2008?

\hypertarget{vsl---policy-implications}{%
\paragraph{VSL - Policy Implications}\label{vsl---policy-implications}}

\begin{itemize}
\item
  Cost per life saved ranges from \$100,000 for unvented space heaters
  to \$72 billion for a proposed standard to reduce occupational
  exposure to formaldehyde.
\item
  Assuming that the figures are estimated correctly, how is economic
  efficiency maximized? (first equimarginal principle)
\item
  Policies only when MB \textgreater{} MC
\item
  Should the government strive to maximize economic efficiency?
\end{itemize}

\hypertarget{valuing-benefits---methods-1}{%
\paragraph{\#3: Valuing Benefits -
Methods}\label{valuing-benefits---methods-1}}

\begin{itemize}
\item
  Stated Preference Method -- valuing based on a survey where the
  respondents state their value
\item
  Contingent Valuation: Asking through a survey, what value respondents
  place on preserving a species, for example.
\item
  Problems involved with Stated Preference
\item
  Strategic bias
\item
  Information bias
\item
  Starting-point bias
\item
  Hypothetical bias
\item
  WTP vs WTA
\item
  ``warm glow'' effect
\item
  Embedding bias -- context of problem
\end{itemize}

\hypertarget{stated-preference-contingent-valuation}{%
\paragraph{Stated Preference -- Contingent
Valuation}\label{stated-preference-contingent-valuation}}

\begin{itemize}
\tightlist
\item
  Economists disagree about the reliability of CV analysis. But doing a
  `state of the art' job can be very expensive
\item
  Analyses of Exxon Valdez oil spill: \$3 million
\item
  Estimated lost passive use value: \$2.8 billion
\item
  CVs provide the only available means for estimating nonmarket benefits
  based primarily on existence value, so they are widely used
\end{itemize}

\hypertarget{discounting-and-npv}{%
\paragraph{\#6-\#7: Discounting and NPV}\label{discounting-and-npv}}

\begin{itemize}
\tightlist
\item
  Office of Management \& Budget 10-year (whitehouse.gov/omb/) 10\%
  (1992) 7\% (2013) 2\%
\item
  Congressional Budget Office: 2\%
\item
  EPA: 5.875\%
\item
  EPA Analysis of Waxman-Markey Bill: 5\%
\item
  Stern Review: 1.4\%
\end{itemize}

\[NPV=\sum_{n=0}^N\frac{NB_n}{(1+n)^n}= \frac{NB}i\]

\hypertarget{an-alternative-cost-effectiveness-analysis}{%
\paragraph{An Alternative -- Cost-Effectiveness
Analysis}\label{an-alternative-cost-effectiveness-analysis}}

\begin{itemize}
\tightlist
\item
  Establish objective (i.e.~how much pollution control?)
\item
  Optimize (i.e.~choose the most cost-effective options given the
  objective)
\end{itemize}

\hypertarget{v.-supply-of-transportation}{%
\subsection{V. Supply of
Transportation}\label{v.-supply-of-transportation}}

\hypertarget{lr-vs-sr-costs}{%
\paragraph{LR vs SR Costs}\label{lr-vs-sr-costs}}

\begin{itemize}
\tightlist
\item
  Short-run (SR) costs
\item
  Some inputs are fixed -- machinery, land, infrastructure
\item
  Other inputs are variable -- labor, materials
\item
  Long-run (LR) costs: all inputs are variable
\end{itemize}

\hypertarget{average-costs-economies-of-scale}{%
\paragraph{Average Costs \& Economies of
Scale}\label{average-costs-economies-of-scale}}

\begin{itemize}
\tightlist
\item
  LR average cost (LAC) is the envelop of SR average costs
\end{itemize}

Economies of scale occurs when MC \textless{} AC; AC is falling

\hypertarget{firm-theory}{%
\paragraph{Firm Theory}\label{firm-theory}}

\texttt{p.64-67} - Inputs: K (capital), L (labor) - Output: y = f(K,L) -
Costs: C = rK + wL, where w is wage and r is rent - Production
isoquants: increasing in inputs, quasi-concave - Firms are
profitmaximizing * In other words, they are also cost-minimizing given a
production level (and prices for inputs)

\hypertarget{firm-production-function}{%
\paragraph{Firm production function}\label{firm-production-function}}

\begin{itemize}
\item
  Cobb-Douglas Production Function
  \[y = f(K,L) = AK^{\alpha}L^{\beta}F^e (\alpha, \beta > 0)\begin{cases}\text{increasing returns to scale, if}&\alpha+\beta > 1\\ \text{constant returns to scale, if}&\alpha+\beta = 1 \\ \text{decreasing returns to scale, if}&\alpha+\beta < 1 \end{cases}\]
  where y measures the amount of electricity produced, K measures
  capital input, L measures labor, and F measures fuel, and where A, a,
  b, and e are constants.
\item
  Corresponding cost function:
\end{itemize}

\[C = By^\frac{1}{(a + b + e)} (P_K)^\frac{a}{(a+b+e)} (P_L)^\frac{b}{(a + b + e)} (P_F)^\frac{e}{(a+b+e)}\]

\hypertarget{firm-cost-function}{%
\paragraph{Firm Cost Function}\label{firm-cost-function}}

\begin{itemize}
\tightlist
\item
  Taking natural log of both sides:
\end{itemize}

\[\ln C = d_0 + d_1 \ln y + d_2\ln{P_K} + d_3\ln{P_L} + d_4\ln{P_F}\]

, where
\(d_0 =\ln B, d_1 =\frac{1}{(a + b + e)}, d_2 =\frac{a}{(a + b + e)}, d_3 =\frac{b}{(a + b + e)}, d_4 = \frac{e}{(a+b+e)}= (1-d_2-d_3)\).
The coefficient d1 is the elasticity of cost with respect to output. It
shows the percentage change in total cost that will be incurred if the
level of output is increased by 1 percent.

\begin{itemize}
\tightlist
\item
  Constant elasticity of cost for each input
\item
  Elasticity of substitution = 1
\item
  Constant share of inputs
\end{itemize}

\hypertarget{example--caves-christensen-and-tretheway1983}{%
\paragraph{Example -Caves, Christensen and
Tretheway(1983)}\label{example--caves-christensen-and-tretheway1983}}

\texttt{p.78\ An\ Illustration\ of\ Economies\ of\ Density\ and\ Size}
\textbf{Long-run Cost Function}

\[\ln C=\text{constance}+0.836\ln Y+0.131\ln N-0.135\ln D-0.277\ln G+0.356\ln w+0.166\ln f+0.478\ln r+\text{time and firm dummies below}\]

\begin{itemize}
\tightlist
\item
  How do factor prices affect cost?
\end{itemize}

\begin{longtable}[]{@{}lllllllll@{}}
\toprule
\begin{minipage}[b]{0.08\columnwidth}\raggedright
\strut
\end{minipage} & \begin{minipage}[b]{0.08\columnwidth}\raggedright
\strut
\end{minipage} & \begin{minipage}[b]{0.08\columnwidth}\raggedright
Y\strut
\end{minipage} & \begin{minipage}[b]{0.08\columnwidth}\raggedright
N\strut
\end{minipage} & \begin{minipage}[b]{0.08\columnwidth}\raggedright
D\strut
\end{minipage} & \begin{minipage}[b]{0.08\columnwidth}\raggedright
G\strut
\end{minipage} & \begin{minipage}[b]{0.08\columnwidth}\raggedright
w\strut
\end{minipage} & \begin{minipage}[b]{0.08\columnwidth}\raggedright
f\strut
\end{minipage} & \begin{minipage}[b]{0.08\columnwidth}\raggedright
r\strut
\end{minipage}\tabularnewline
\midrule
\endhead
\begin{minipage}[t]{0.08\columnwidth}\raggedright
\strut
\end{minipage} & \begin{minipage}[t]{0.08\columnwidth}\raggedright
\strut
\end{minipage} & \begin{minipage}[t]{0.08\columnwidth}\raggedright
output\strut
\end{minipage} & \begin{minipage}[t]{0.08\columnwidth}\raggedright
\# of points\strut
\end{minipage} & \begin{minipage}[t]{0.08\columnwidth}\raggedright
stage length\strut
\end{minipage} & \begin{minipage}[t]{0.08\columnwidth}\raggedright
load factor\strut
\end{minipage} & \begin{minipage}[t]{0.08\columnwidth}\raggedright
wage rate\strut
\end{minipage} & \begin{minipage}[t]{0.08\columnwidth}\raggedright
fuel price\strut
\end{minipage} & \begin{minipage}[t]{0.08\columnwidth}\raggedright
cost of capital material\strut
\end{minipage}\tabularnewline
\bottomrule
\end{longtable}

\begin{itemize}
\tightlist
\item
  What are shares of total cost for each factor?
\item
  Economies of scale (output)
\item
  Economies of density
\item
  Increase in output (Y), but not \# of points (N)
\item
  Economies of size
\item
  Increase in both output (Y) and \# of points (N)
\item
  Economies of scope?
\item
  If y1 is passenger and y2 is freight, then economies of scope occur
  when: C(y1,y2) \textless{} C(y1,0) + C(0,y2).
\item
  Cost-reducing technological change over time?
\end{itemize}

Finding 1: How will increases in factor prices affect total costs?

Finding 2: What does the cost study indicate about the share of total
costs expended on the various factors of production?

Finding 3: What does the study indicate about economies of density?

Finding 4: What does the study indicate about economies of size?

Finding 5: Is there evidence of cost-reducing technical change over
time?


\end{document}
